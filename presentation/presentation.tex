%%%%%%%%%%%%%%%%%%%%%%%%%%%%%%%%%%%%%%%%%
% Beamer Presentation
% LaTeX Template
% Version 1.0 (10/11/12)
%
% This template has been downloaded from:
% http://www.LaTeXTemplates.com
%
% License:
% CC BY-NC-SA 3.0 (http://creativecommons.org/licenses/by-nc-sa/3.0/)
%
%%%%%%%%%%%%%%%%%%%%%%%%%%%%%%%%%%%%%%%%%

%----------------------------------------------------------------------------------------
%	PACKAGES AND THEMES
%----------------------------------------------------------------------------------------

\documentclass{beamer}
\usepackage{amssymb}
\usepackage{amsmath,bm}
\graphicspath{ {pics/} }
\usepackage{tikz}
\tikzset{
  every overlay node/.style={
    draw=black,fill=white,rounded corners,anchor=north west,
  },
}
% Usage:
% \tikzoverlay at (-1cm,-5cm) {content};
% or
% \tikzoverlay[text width=5cm] at (-1cm,-5cm) {content};
\def\tikzoverlay{%
   \tikz[baseline,overlay]\node[every overlay node]
}%
\usepackage[absolute,overlay]{textpos}
\mode<presentation> {

% The Beamer class comes with a number of default slide themes
% which change the colors and layouts of slides. Below this is a list
% of all the themes, uncomment each in turn to see what they look like.

%\usetheme{default}
%\usetheme{AnnArbor}
%\usetheme{Antibes}
%\usetheme{Bergen}
%\usetheme{Berkeley}
%\usetheme{Berlin}
%\usetheme{Boadilla}
%\usetheme{CambridgeUS}
%\usetheme{Copenhagen}
%\usetheme{Darmstadt}
%\usetheme{Dresden}
%\usetheme{Frankfurt}
%\usetheme{Goettingen}
%\usetheme{Hannover}
%\usetheme{Ilmenau}
%\usetheme{JuanLesPins}
\usetheme{Luebeck}
%\usetheme{Madrid}
%\usetheme{Malmoe}
%\usetheme{Marburg}
%\usetheme{Montpellier}
%\usetheme{PaloAlto}
%\usetheme{Pittsburgh}
%\usetheme{Rochester}
%\usetheme{Singapore}
%\usetheme{Szeged}
%\usetheme{Warsaw}

% As well as themes, the Beamer class has a number of color themes
% for any slide theme. Uncomment each of these in turn to see how it
% changes the colors of your current slide theme.

%\usecolortheme{albatross}
%\usecolortheme{beaver}
%\usecolortheme{beetle}
%\usecolortheme{crane}
%\usecolortheme{dolphin}
%\usecolortheme{dove}
%\usecolortheme{fly}
%\usecolortheme{lily}
%\usecolortheme{orchid}
%\usecolortheme{rose}
%\usecolortheme{seagull}
%\usecolortheme{seahorse}
\usecolortheme{whale}
%\usecolortheme{wolverine}
\setbeamertemplate{itemize items}[ball]
%\setbeamertemplate{footline} % To remove the footer line in all slides uncomment this line
\setbeamertemplate{footline}[page number] % To replace the footer line in all slides with a simple slide count uncomment this line

%\setbeamertemplate{navigation symbols}{} % To remove the navigation symbols from the bottom of all slides uncomment this line
}

\usepackage{graphicx} % Allows including images
\usepackage{booktabs} % Allows the use of \toprule, \midrule and \bottomrule in tables

%----------------------------------------------------------------------------------------
%	TITLE PAGE
%----------------------------------------------------------------------------------------

\title[Short title]{Analyses of Convolution Neural Networks for automatic tagging of music tracks } % The short title appears at the bottom of every slide, the full title is only on the title page

\author{Aravind Sankaran} % Your name
\institute[RWTH Aachen] % Your institution as it will appear on the bottom of every slide, may be shorthand to save space
{
RWTH Aachen \\ % Your institution for the title page
\medskip
\textit{aravind.sankaran@rwth-aachen.de} % Your email address
}
\date{\today} % Date, can be changed to a custom date

\begin{document}

\begin{frame}
\titlepage % Print the title page as the first slide
\end{frame}

\begin{frame}
\frametitle{Acknowledgement}
\begin{itemize}
\item Prof. Paolo Bientinesi
\item Prof. Marco Alunno
\end{itemize}

\end{frame}
\begin{frame}
\frametitle{AIM}
\begin{picture}(50,50)
\put(20,-100){\hbox{\includegraphics[scale=0.4]{aim}}} 
\end{picture} 
\end{frame}

\begin{frame}
\frametitle{AIM}
\begin{picture}(50,50)
\put(20,-100){\hbox{\includegraphics[scale=0.35]{aim_2}}} 
\end{picture} 
\end{frame}

\begin{frame}
\frametitle{APPLICATION}
\begin{block}{\textbf{User specific recommendation system?}}
$\mathbf{\rightarrow}$ \textbf{Do you have to create a dataset with ground-truth?}\\
\pause
\qquad $\mathbf{\rightarrow}$ \textit{Lets look at some solutions where you \textbf{don't} have to..}

\end{block}
\bigskip

\textbf{Collaborative filtering :}
\begin{itemize}
\item {\color<2>[RGB]{0,100,0}Exploits social trends}
\item {\color<2>[RGB]{0,100,0}No information from audio content is used}
\item {\color{red} Cold-Start Problem}
\end{itemize}
\end{frame}

\begin{frame}
\frametitle{APPLICATION}
\begin{block}{\textbf{User specific recommendation system?}}
$\mathbf{\rightarrow}$ \textbf{Do you have to create a dataset with ground-truth?}\\
\qquad $\mathbf{\rightarrow}$ \textit{Lets look at some solutions where you \textbf{don't} have to..}
\end{block}
\bigskip

\textbf{Collaborative + Content-based :}
\begin{itemize}
\item {\color<1>[RGB]{0,100,0}Gather training data by crowd sourcing}
\item {\color<1>[RGB]{0,100,0}User specific recommendations by filtering popular tags}
\end{itemize}
\end{frame}

\begin{frame}
\frametitle{APPLICATION}
\begin{block}{\textbf{User specific recommendation system?}}
$\mathbf{\rightarrow}$ \textbf{{\color{red} When} do you have to create a dataset with ground-truth?}
\end{block}
\end{frame}

\begin{frame}
\frametitle{APPLICATION}
\begin{block}{\textbf{Recommendation system for {\color{green} experts}?}}
$\mathbf{\rightarrow}$ \textbf{{\color{red} When} do you have to create a dataset with ground-truth?}\\
\qquad $\mathbf{\rightarrow}$ \textit{Lets look from an artist's point of view ..}

\end{block}
\end{frame}

\begin{frame}
\frametitle{PIPELINE}
\begin{picture}(50,50)
\put(90,-100){\hbox{\includegraphics[scale=0.4]{gpipe}}} 
\end{picture} 
\end{frame}

\begin{frame}
\frametitle{PIPELINE}
\begin{columns}[c]
\column{.45\textwidth}
\begin{picture}(50,50)
\put(0,-100){\hbox{\includegraphics[scale=0.4]{gpipe_1}}} 
\end{picture} 
%\hspace{.45\textwidth}
%\vrule{}
\column{.5\textwidth}
\begin{block}{\textbf{Inputs :}}
\begin{itemize}
\item {\color<1>[RGB]{0,100,0}Digital Signal (.mp3, .wav)}
\item Sheets of musical notes
\end{itemize}
\end{block}
\end{columns}
\end{frame}

\begin{frame}
\frametitle{PIPELINE}
\begin{columns}[c]
\column{.45\textwidth}
\begin{picture}(50,50)
\put(0,-100){\hbox{\includegraphics[scale=0.4]{gpipe_3}}} 
\end{picture} 
%\hspace{.45\textwidth}
%\vrule{}
\column{.5\textwidth}
\begin{block}{\textbf{Feature Extraction :}}
\begin{itemize}
\item $\mathbb{R}^{N} \rightarrow \mathbb{R}^{T} \quad T < N$
\item \textbf{Organized :}\\ Encode information about discriminants
\item \textbf{Robust :}\\ Transformation should be well posed
\end{itemize}
\end{block}
\end{columns}
\end{frame}

\begin{frame}
\frametitle{PIPELINE}
\begin{picture}(50,50)
\put(40,-100){\hbox{\includegraphics[scale=0.4]{gpipe_4}}} 
\end{picture} 
\end{frame}

\begin{frame}
\frametitle{REPRESENTATION}
\begin{columns}[c]
\column{.45\textwidth}
\begin{figure}
\includegraphics[width=\textwidth]{rep}
\end{figure}
\column{.5\textwidth}
\[
\textbf{a} = (a_{1}, a_{2}, .. a_{N}) = a_{1}\textbf{e}_{1} + ... a_{N}\textbf{e}_{N}
\]
\pause
\begin{block}{\textbf{Basis}}
A group of vectors forms a basis of a vector space $\mathbb{V}$ if every vector in $\mathbb{V}$ can be represented as a linear combination of the basis vectors
\end{block}
\end{columns}
\end{frame}

\begin{frame}
\frametitle{REPRESENTATION}
\begin{columns}[c]
\column{.45\textwidth}
\begin{figure}
\includegraphics[width=\textwidth]{rep}
\end{figure}
\column{.5\textwidth}
\[
\textbf{a} = a_{1}\textbf{e}_{1} + ... a_{N}\textbf{e}_{N} = \mathbb{I}_{N}\textbf{a}
\]
\[
\textbf{a} = c_{1}\textbf{q}_{1} + ... c_{M}\textbf{q}_{M} = \textbf{Q}\textbf{c}
\]
\pause
\[
\textbf{Q}^{-1}\textbf{a} = \textbf{c} \qquad \textbf{Q}^{-1} \in \mathbb{R}^{M \times N}
\]
\end{columns}
\end{frame}

\begin{frame}
\frametitle{REPRESENTATION}
\begin{columns}[c]
\column{.45\textwidth}
\begin{figure}
\includegraphics[width=\textwidth]{rep}
\end{figure}
\column{.5\textwidth}
\[
\textbf{Q}^{-1}\textbf{a} = \textbf{c} \qquad \textbf{Q}^{-1} \in \mathbb{R}^{M \times N}
\]
\begin{block}{\textbf{Exponential Fourier Theorem}}
Complex exponentials which are functions of frequencies form basis for \textit{periodic} function
\end{block}
\pause
\begin{block}{\textbf{Fourier Transform}}
Application of \textit{Fourier Theorem} for general signals.
\[
\textbf{Q}^{-1}[i] = \textbf{e}^{-j \omega t} \qquad i \in \{0,1..,M\}
\]
\end{block}
\end{columns}
\end{frame}

\begin{frame}[noframenumbering]
\frametitle{REPRESENTATION}
\begin{columns}[c]
\column{.45\textwidth}
\begin{figure}
\includegraphics[width=\textwidth]{rep_1}
\end{figure}
\column{.5\textwidth}
\[
\textbf{Q}^{-1}\textbf{a} = \textbf{c} \qquad \textbf{Q}^{-1} \in \mathbb{R}^{M \times N}
\]
\begin{block}{\textbf{Fourier Transform}}
Application of \textit{Fourier Theorem} for general signals.
\[
\textbf{Q}^{-1}[i] = \textbf{e}^{-j \omega t} \qquad i \in \{0,1..,M\}
\]
\end{block}
\pause
\begin{block}{\textbf{Short-time Fourier Transform}}
\[
\textbf{C} = \textbf{a} \bm{\star} \textbf{Q}^{-1} \qquad \textbf{C} \in \mathbb{C}^{M \times P}
\]
\end{block}
\end{columns}
\end{frame}

\begin{frame}[noframenumbering]
\frametitle{REPRESENTATION}
\begin{columns}[c]
\column{.45\textwidth}
\begin{figure}
\includegraphics[width=\textwidth]{rep_1}
\end{figure}
\column{.5\textwidth}
\begin{block}{\textbf{Short-time Fourier Transform}}
\[
\textbf{C} = \textbf{a} \bm{\star} \textbf{Q}^{-1} \qquad \textbf{C} \in \mathbb{C}^{M \times P}
\]
\end{block}
\begin{block}{\textbf{Fast Fourier Transform}}
Faster version of STFT that exploits the symmetry of sinusoids.\\
\textbf{STFT :} $O(N^{2})$\\
\textbf{FFT :} $O(NlogN)$
\end{block}
\end{columns}
\end{frame}

\begin{frame}
\frametitle{REPRESENTATION}
\begin{columns}[c]
\column{.45\textwidth}
\begin{figure}
\includegraphics[width=\textwidth]{rep_2}
\end{figure}
\column{.5\textwidth}
\[
\textbf{C} = \textbf{a} \bm{\star} \textbf{Q}^{-1} \qquad \textbf{C} \in \mathbb{C}^{M \times P}
\]
\begin{block}{\textbf{Spectrogram representations}}
\begin{itemize}
\item \textbf{Mel Spec :}  \\ $\textbf{R} = \textbf{M}. \textbf{C} \quad \vee \textbf{M} \in \mathbb{R}^{R \times M}$
\item \textbf{Chromagram :}\\$\textbf{R} = \textbf{C}.\textbf{M}_{C}$
\item \textbf{Tempogram :}\\$\textbf{R} = \textbf{C} \star \textbf{M}_{T}$  
\end{itemize}
\end{block}
\end{columns}
\end{frame}

\begin{frame}
\frametitle{DIMENSIONALITY REDUCTION}
\begin{columns}[c]
\column{.45\textwidth}
\begin{figure}
\includegraphics[width=\textwidth]{rep_3}
\end{figure}
\column{.5\textwidth}
\[
\textbf{C} = \textbf{a} \bm{\star} \textbf{Q}^{-1} \qquad \textbf{C} \in \mathbb{C}^{M \times P}
\]
\[
\textbf{R} = \textbf{C} \bm{\star} \textbf{M} \qquad \textbf{R} \in \mathbb{R}^{R \times P}
\]
\begin{block}{\textbf{Principal Component Analysis}}
Represent $\textbf{R}$ in a basis that is a function of variance in the information.
\end{block}
\end{columns}
\end{frame}

\begin{frame}
\frametitle{DIMENSIONALITY REDUCTION}
\begin{columns}[c]
\column{.45\textwidth}
\begin{figure}
\includegraphics[width=\textwidth]{dim_1}
\end{figure}
\column{.5\textwidth}
\[
\textbf{C} = \textbf{a} \bm{\star} \textbf{Q}^{-1} \qquad \textbf{C} \in \mathbb{C}^{M \times P}
\]
\[
\textbf{R} = \textbf{C} \bm{\star} \textbf{M} \qquad \textbf{R} \in \mathbb{R}^{R \times P}
\]
\begin{block}{\textbf{Principal Component Analysis}}
Represent $\textbf{R}$ in a basis that is a function of variance in the information.\\
$\textbf{\^R} = Center(\textbf{R})$\\
$\bm{\Sigma} = \frac{1}{P}\textbf{\^R} \textbf{\^R}^{T}$\\
$\textbf{V}\bm{\Lambda}\textbf{V}^{T} = \bm{\Sigma}$\\
$\textbf{X} = Truncate(\textbf{V}^{T})\textbf{\^R}$
\end{block}
\end{columns}
\end{frame}

\begin{frame}
\frametitle{DIMENSIONALITY REDUCTION}
\begin{columns}[c]
\column{.45\textwidth}
\begin{figure}
\includegraphics[width=\textwidth]{dim_1}
\end{figure}
\column{.5\textwidth}
\[
\textbf{C} = \textbf{a} \bm{\star} \textbf{Q}^{-1} \qquad \textbf{C} \in \mathbb{C}^{M \times P}
\]
\[
\textbf{R} = \textbf{C} \bm{\star} \textbf{M} \qquad \textbf{R} \in \mathbb{R}^{R \times P}
\]
\begin{block}{\textbf{Mel-Frequency Cepstral Coefficients}}
Basis functions of \textit{principal components} of log spectra are very similiar to \textit{cosine transform}\\
$\textbf{V}^{T}[i] = cos( \omega t) \quad i \in \{0,1..,T\}$
\end{block}
\end{columns}
\end{frame}

\begin{frame}
\frametitle{DIMENSIONALITY REDUCTION}
\begin{columns}[c]
\column{.45\textwidth}
\begin{figure}
\includegraphics[width=\textwidth]{dim_1}
\end{figure}
\column{.5\textwidth}
\[
\textbf{C} = \textbf{a} \bm{\star} \textbf{Q}^{-1} \qquad \textbf{C} \in \mathbb{C}^{M \times P}
\]
\[
\textbf{R} = \textbf{C} \bm{\star} \textbf{M} \qquad \textbf{R} \in \mathbb{R}^{R \times P}
\]
\begin{block}{\textbf{Mel-Frequency Cepstral Coefficients}}
Basis functions of \textit{principal components} of log spectra are very similiar to \textit{cosine transform}\\
$\textbf{V}^{T}[i] = cos( \omega t) \quad i \in \{0,1..,T\}$
\end{block}
\end{columns}
\end{frame}

\begin{frame}
\frametitle{DIMENSIONALITY REDUCTION}
\begin{columns}[c]
\column{.45\textwidth}
\begin{figure}
\includegraphics[width=\textwidth]{dim_2}
\end{figure}
\column{.5\textwidth}
$\textbf{C} = \textbf{a} \bm{\star} \textbf{Q}^{-1} \qquad \textbf{C} \in \mathbb{C}^{M \times P}$\\
$\textbf{R} = \textbf{C} \bm{\star} \textbf{M} \qquad \textbf{R} \in \mathbb{R}^{R \times P}$
$\textbf{X} = \textbf{R} \bm{\star} \textbf{V}^{T} \qquad \textbf{X} \in \mathbb{R}^{T \times P}$\\
\begin{block}{\textbf{Mel-Frequency Cepstral Coefficients}}
Basis functions of \textit{principal components} of log spectra are very similiar to \textit{cosine transform}\\
$\textbf{V}^{T}[i] = cos( \omega t) \quad i \in \{0,1..,T\}$
\end{block}
\end{columns}
\end{frame}

\begin{frame}
\frametitle{TEMPORAL APPROXIMATION}
\begin{columns}[c]
\column{.45\textwidth}
\begin{figure}
\includegraphics[width=\textwidth]{tmp_1}
\end{figure}
\column{.5\textwidth}
$\textbf{C} = \textbf{a} \bm{\star} \textbf{Q}^{-1} \qquad \textbf{C} \in \mathbb{C}^{M \times P}$\\
$\textbf{R} = \textbf{C} \bm{\star} \textbf{M} \qquad \textbf{R} \in \mathbb{R}^{R \times P}$
$\textbf{X} = \textbf{R} \bm{\star} \textbf{V}^{T} \qquad \textbf{X} \in \mathbb{R}^{T \times P}$\\
\end{columns}
\end{frame}

\begin{frame}
\frametitle{TEMPORAL APPROXIMATION}
\begin{columns}[c]
\column{.45\textwidth}
\begin{figure}
\includegraphics[width=\textwidth]{tmp_2}
\end{figure}
\column{.5\textwidth}
$\textbf{C} = \textbf{a} \bm{\star} \textbf{Q}^{-1} \qquad \textbf{C} \in \mathbb{C}^{M \times P}$\\
$\textbf{R} = \textbf{C} \bm{\star} \textbf{M} \qquad \textbf{R} \in \mathbb{R}^{R \times P}$
$\textbf{X} = \textbf{R} \bm{\star} \textbf{V}^{T} \qquad \textbf{X} \in \mathbb{R}^{T \times P}$\\
\begin{block}{\textbf{Bag Of Frames}}
\begin{itemize}
\item Assign each column of $\textbf{X}$ to the nearest of $K$ clusters.
\item Count the number of assignments to each of the $K$ clusters.
\item The resulting feature is of dimension $K$
\end{itemize}
\end{block}
\end{columns}
\end{frame}

\begin{frame}
\frametitle{TEMPORAL APPROXIMATION}
\textbf{K - Means}[1] :
\begin{figure}
\includegraphics[width=0.8\textwidth]{kmeans}
\end{figure}
\end{frame}

\begin{frame}
\frametitle{CLASSIFICATION}
\textbf{Feature Extraction :} $\mathbb{R}^{N_{f}} \rightarrow \mathbb{R}^{K}$
\bigskip

$\quad \textbf{C} = \textbf{a} \bm{\star} \textbf{Q}^{-1} \qquad \textbf{C} \in \mathbb{C}^{M \times P}$\\
$\quad \textbf{R} = \textbf{C} \bm{\star} \textbf{M} \qquad \textbf{R} \in \mathbb{R}^{R \times P}$\\
$\quad \textbf{X} = \textbf{R} \bm{\star} \textbf{V}^{T} \qquad \textbf{X} \in \mathbb{R}^{T \times P}$\\
$\quad \textbf{f} = Temporal\_Approx(\textbf{X}) \qquad \textbf{f} \in \mathbb{R}^{K}$\\
\bigskip

\textbf{Classification :} $\mathbb{R}^{K} \rightarrow \mathbb{R}^{L}$
\bigskip

$\quad \textbf{y} = \bm{\Phi}(\textbf{f}) \qquad \textbf{y} \in \mathbb{R}^{L}$
\end{frame}

\begin{frame}
\frametitle{CLASSIFICATION}
\begin{columns}[c]
\column{.45\textwidth}
\textbf{Single-layer perceptron :}
\bigskip

$\quad \textbf{y} = \textbf{W}\textbf{f} \qquad \textbf{W} \in \mathbb{R}^{L \times K}$

\column{.5\textwidth}
\begin{block}{\textbf{FE :} $\mathbb{R}^{N_{f}} \rightarrow \mathbb{R}^{K}$}

$\textbf{C} = \textbf{a} \bm{\star} \textbf{Q}^{-1} \qquad \textbf{C} \in \mathbb{C}^{M \times P}$\\
$\textbf{R} = \textbf{C} \bm{\star} \textbf{M} \qquad \textbf{R} \in \mathbb{R}^{R \times P}$\\
$\textbf{X} = \textbf{R} \bm{\star} \textbf{V}^{T} \qquad \textbf{X} \in \mathbb{R}^{T \times P}$\\
$\textbf{f} = T(\textbf{X}) \qquad \textbf{f} \in \mathbb{R}^{K}$\\
\end{block}
\begin{block}{\textbf{Classification :} $\mathbb{R}^{K} \rightarrow \mathbb{R}^{L}$}
$\quad \textbf{y} = \bm{\Phi}(\textbf{f}) \qquad \textbf{y} \in \mathbb{R}^{L}$
\end{block}
\end{columns}
\end{frame}

\begin{frame}
\frametitle{CLASSIFICATION}
\begin{columns}[c]
\column{.45\textwidth}
\textbf{Single-layer perceptron :}
\bigskip

$\quad \textbf{y} = \textbf{{\color{red}W}}\textbf{f} \qquad \textbf{W} \in \mathbb{R}^{L \times K}$\\
\bigskip

Solve for \textbf{{\color{red}W}} with the training data

\column{.5\textwidth}
\begin{block}{\textbf{FE :} $\mathbb{R}^{N_{f}} \rightarrow \mathbb{R}^{K}$}

$\textbf{C} = \textbf{a} \bm{\star} \textbf{Q}^{-1} \qquad \textbf{C} \in \mathbb{C}^{M \times P}$\\
$\textbf{R} = \textbf{C} \bm{\star} \textbf{M} \qquad \textbf{R} \in \mathbb{R}^{R \times P}$\\
$\textbf{X} = \textbf{R} \bm{\star} \textbf{V}^{T} \qquad \textbf{X} \in \mathbb{R}^{T \times P}$\\
$\textbf{f} = T(\textbf{X}) \qquad \textbf{f} \in \mathbb{R}^{K}$\\
\end{block}
\begin{block}{\textbf{Classification :} $\mathbb{R}^{K} \rightarrow \mathbb{R}^{L}$}
$\quad \textbf{y} = \bm{\Phi}(\textbf{f}) \qquad \textbf{y} \in \mathbb{R}^{L}$
\end{block}
\end{columns}
\end{frame}

\begin{frame}
\frametitle{CLASSIFICATION}
\begin{columns}[c]
\column{.45\textwidth}
\textbf{Two-layer perceptron :}
\bigskip

$\quad \textbf{y} = \sigma(\textbf{W}_{2}ReLU({\color{red}\textbf{W}_{1}\textbf{f}}))$
\bigskip

$\quad \textbf{W}_{2} \in \mathbb{R}^{L \times H} \quad \textbf{W}_{1} \in \mathbb{R}^{H \times K}$

\column{.5\textwidth}
\begin{block}{\textbf{FE :} $\mathbb{R}^{N_{f}} \rightarrow \mathbb{R}^{K}$}

$\textbf{C} = \textbf{a} \bm{\star} \textbf{Q}^{-1} \qquad \textbf{C} \in \mathbb{C}^{M \times P}$\\
$\textbf{R} = \textbf{C} \bm{\star} \textbf{M} \qquad \textbf{R} \in \mathbb{R}^{R \times P}$\\
$\textbf{X} = \textbf{R} \bm{\star} \textbf{V}^{T} \qquad \textbf{X} \in \mathbb{R}^{T \times P}$\\
$\textbf{f} = T(\textbf{X}) \qquad \textbf{f} \in \mathbb{R}^{K}$\\
\end{block}
\begin{block}{\textbf{Classification :} $\mathbb{R}^{K} \rightarrow \mathbb{R}^{L}$}
$\quad \textbf{y} = \bm{\Phi}(\textbf{f}) \qquad \textbf{y} \in \mathbb{R}^{L}$
\end{block}
\end{columns}
\end{frame}


\begin{frame}
\frametitle{CLASSIFICATION}
\begin{columns}[c]
\column{.45\textwidth}
$\quad \textbf{y} = \sigma({\color{red}\textbf{W}_{2}}ReLU({\color{red}\textbf{W}_{1}}\textbf{f}))$
\bigskip

\noindent \textbf{Training :}

\column{.5\textwidth}
\begin{block}{\textbf{FE :} $\mathbb{R}^{N_{f}} \rightarrow \mathbb{R}^{K}$}

$\textbf{C} = \textbf{a} \bm{\star} \textbf{Q}^{-1} \qquad \textbf{C} \in \mathbb{C}^{M \times P}$\\
$\textbf{R} = \textbf{C} \bm{\star} \textbf{M} \qquad \textbf{R} \in \mathbb{R}^{R \times P}$\\
$\textbf{X} = \textbf{R} \bm{\star} \textbf{V}^{T} \qquad \textbf{X} \in \mathbb{R}^{T \times P}$\\
$\textbf{f} = T(\textbf{X}) \qquad \textbf{f} \in \mathbb{R}^{K}$\\
\end{block}
\begin{block}{\textbf{Classification :} $\mathbb{R}^{K} \rightarrow \mathbb{R}^{L}$}
$\textbf{y} = \sigma(\textbf{W}_{2}ReLU(\textbf{W}_{1}\textbf{f})) \quad \textbf{y} \in \mathbb{R}^{L}$
\end{block}
\end{columns}
\end{frame}


\begin{frame}
\frametitle{Overview} % Table of contents slide, comment this block out to remove it
\tableofcontents % Throughout your presentation, if you choose to use \section{} and \subsection{} commands, these will automatically be printed on this slide as an overview of your presentation
\end{frame}

%----------------------------------------------------------------------------------------
%	PRESENTATION SLIDES
%----------------------------------------------------------------------------------------

%------------------------------------------------
\section{First Section} % Sections can be created in order to organize your presentation into discrete blocks, all sections and subsections are automatically printed in the table of contents as an overview of the talk
%------------------------------------------------

\subsection{Subsection Example} % A subsection can be created just before a set of slides with a common theme to further break down your presentation into chunks

\begin{frame}
\frametitle{Paragraphs of Text}
Sed iaculis dapibus gravida. Morbi sed tortor erat, nec interdum arcu. Sed id lorem lectus. Quisque viverra augue id sem ornare non aliquam nibh tristique. Aenean in ligula nisl. Nulla sed tellus ipsum. Donec vestibulum ligula non lorem vulputate fermentum accumsan neque mollis.\\~\\

Sed diam enim, sagittis nec condimentum sit amet, ullamcorper sit amet libero. Aliquam vel dui orci, a porta odio. Nullam id suscipit ipsum. Aenean lobortis commodo sem, ut commodo leo gravida vitae. Pellentesque vehicula ante iaculis arcu pretium rutrum eget sit amet purus. Integer ornare nulla quis neque ultrices lobortis. Vestibulum ultrices tincidunt libero, quis commodo erat ullamcorper id.
\end{frame}

%------------------------------------------------

\begin{frame}
\frametitle{Bullet Points}
\begin{itemize}
\item Lorem ipsum dolor sit amet, consectetur adipiscing elit
\item Aliquam blandit faucibus nisi, sit amet dapibus enim tempus eu
\item Nulla commodo, erat quis gravida posuere, elit lacus lobortis est, quis porttitor odio mauris at libero
\item Nam cursus est eget velit posuere pellentesque
\item Vestibulum faucibus velit a augue condimentum quis convallis nulla gravida
\end{itemize}
\end{frame}

%------------------------------------------------

\begin{frame}
\frametitle{Blocks of Highlighted Text}
\begin{block}{Block 1}
Lorem ipsum dolor sit amet, consectetur adipiscing elit. Integer lectus nisl, ultricies in feugiat rutrum, porttitor sit amet augue. Aliquam ut tortor mauris. Sed volutpat ante purus, quis accumsan dolor.
\end{block}

\begin{block}{Block 2}
Pellentesque sed tellus purus. Class aptent taciti sociosqu ad litora torquent per conubia nostra, per inceptos himenaeos. Vestibulum quis magna at risus dictum tempor eu vitae velit.
\end{block}

\begin{block}{Block 3}
Suspendisse tincidunt sagittis gravida. Curabitur condimentum, enim sed venenatis rutrum, ipsum neque consectetur orci, sed blandit justo nisi ac lacus.
\end{block}
\end{frame}

%------------------------------------------------

\begin{frame}
\frametitle{Multiple Columns}
\begin{columns}[c] % The "c" option specifies centered vertical alignment while the "t" option is used for top vertical alignment

\column{.45\textwidth} % Left column and width
\textbf{Heading}
\begin{enumerate}
\item Statement
\item Explanation
\item Example
\end{enumerate}

\column{.5\textwidth} % Right column and width
Lorem ipsum dolor sit amet, consectetur adipiscing elit. Integer lectus nisl, ultricies in feugiat rutrum, porttitor sit amet augue. Aliquam ut tortor mauris. Sed volutpat ante purus, quis accumsan dolor.

\end{columns}
\end{frame}

%------------------------------------------------
\section{Second Section}
%------------------------------------------------

\begin{frame}
\frametitle{Table}
\begin{table}
\begin{tabular}{l l l}
\toprule
\textbf{Treatments} & \textbf{Response 1} & \textbf{Response 2}\\
\midrule
Treatment 1 & 0.0003262 & 0.562 \\
Treatment 2 & 0.0015681 & 0.910 \\
Treatment 3 & 0.0009271 & 0.296 \\
\bottomrule
\end{tabular}
\caption{Table caption}
\end{table}
\end{frame}

%------------------------------------------------

\begin{frame}
\frametitle{Theorem}
\begin{theorem}[Mass--energy equivalence]
$E = mc^2$
\end{theorem}
\end{frame}

%------------------------------------------------

\begin{frame}[fragile] % Need to use the fragile option when verbatim is used in the slide
\frametitle{Verbatim}
\begin{example}[Theorem Slide Code]
\begin{verbatim}
\begin{frame}
\frametitle{Theorem}
\begin{theorem}[Mass--energy equivalence]
$E = mc^2$
\end{theorem}
\end{frame}\end{verbatim}
\end{example}
\end{frame}

%------------------------------------------------

\begin{frame}
\frametitle{Figure}
Uncomment the code on this slide to include your own image from the same directory as the template .TeX file.
%\begin{figure}
%\includegraphics[width=0.8\linewidth]{test}
%\end{figure}
\end{frame}

%------------------------------------------------

\begin{frame}[fragile] % Need to use the fragile option when verbatim is used in the slide
\frametitle{Citation}
An example of the \verb|\cite| command to cite within the presentation:\\~

This statement requires citation \cite{p1}.
\end{frame}

%------------------------------------------------

\begin{frame}
\frametitle{References}
\footnotesize{
\begin{thebibliography}{99} % Beamer does not support BibTeX so references must be inserted manually as below
\bibitem[Smith, 2012]{p1} John Smith (2012)
\newblock Title of the publication
\newblock \emph{Journal Name} 12(3), 45 -- 678.
\end{thebibliography}
}
\end{frame}

%------------------------------------------------

\begin{frame}
\Huge{\centerline{The End}}
\end{frame}

%----------------------------------------------------------------------------------------

\end{document}
