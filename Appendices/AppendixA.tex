% Appendix A

%\chapter*{Frequently Asked Questions} % Main appendix title

%\label{AppendixA} % For referencing this appendix elsewhere, use \ref{AppendixA}
\begin{appendices}

\chapter{}
\section{Basis Transformation}
Any vector $\textbf{y} = [y_{1}, y_{2},...,y_{n}] \in V$ can be represented as a \gls{linear combination} of it's \gls{basis} $\textbf{V} = [\textbf{v}_{1}, \textbf{v}_{2},...,\textbf{v}_{n}] \in \mathbb{C}^{N \times N}$
  
\[
\textbf{y} = \displaystyle\sum_{i=1}^{N}\alpha_{i}\textbf{v}_{i} = \textbf{V}^{T}\bm{\alpha}
\]
Where $\alpha_{i}$ are the coefficients or coordinates with respect to $\textbf{v}_{i}$. If there exists another set of \gls{basis} vectors $\textbf{U} = [\textbf{u}_{1}, \textbf{u}_{2},...,\textbf{u}_{m}] \in \mathbb{C}^{M \times M}$, then we can represent $\textbf{y}$ as a \gls{linear combination} of $M$ \gls{basis}.
\[
\textbf{y} = \displaystyle\sum_{i=1}^{M}\beta_{i}\textbf{u}_{i} = \textbf{U}^{T}\bm{\beta}
\]
Where $\beta_{i}$ are the coefficients or coordinates with respect to $\textbf{u}_{i}$.
\bigskip

\noindent For a vector in basis $\textbf{V}$ with co-ordinates $\bm{\alpha}$, \textbf{basis transformation} is defined as representing the same vector with the new co-ordinates $\bm{\beta}$ in basis $\textbf{U}$.
\bigskip

\noindent Cartesian basis ($\textbf{e}_{i}$) are standard basis vectors with $\bm{\alpha} = \textbf{y}$
\[
\textbf{y} = \displaystyle\sum_{i=1}^{N}y_{i}\textbf{e}_{i} = \displaystyle\sum_{i=1}^{M}\beta_{i}\textbf{u}_{i}
\]

\end{appendices}