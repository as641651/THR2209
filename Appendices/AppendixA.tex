% Appendix A

%\chapter*{Frequently Asked Questions} % Main appendix title

%\label{AppendixA} % For referencing this appendix elsewhere, use \ref{AppendixA}
\begin{appendices}

\chapter{}
\section{Basis Transformation}

Here we discuss only transformation from standard \gls{basis} or Cartesian \gls{basis}.
\bigskip

\noindent The standard \gls{basis} for $\mathbb{R}^{N}$ is the ordered sequence $\textbf{E}_{n} = [\textbf{e}_{1}, \textbf{e}_{2}, ..,\textbf{e}_{n}]$, where $\textbf{e}_{i}$ is a vector with 1 in $\textit{i}^{th}$ place and 0 elsewhere. Any vector $\textbf{x} = [x_{1}, x_{2},...,x_{n}] \in \mathbb{R}^{N}$ can be represented as a \gls{linear combination} of $\textbf{E}_{n}$ as,
\[
\textbf{x} = \displaystyle\sum_{i=1}^{N}x_{i}\textbf{e}_{i} = \textbf{E}_{n}\textbf{x}
\]

\noindent \textbf{Basis transformation} from standard \gls{basis} is defined as representing the same vector $\textbf{x}$ with the new co-ordinates $[y_{1}, y_{2},...,y_{m}]$ in \gls{basis} $\textbf{V} = [\textbf{v}_{1}, \textbf{v}_{2}, ..,\textbf{v}_{m}] \in \textbf{R}^{N \times M} $.
\bigskip

\[
\textbf{x} = \displaystyle\sum_{i=1}^{M}y_{i}\textbf{v}_{i} = \textbf{V}\textbf{y} \qquad \textbf{y} \in \mathbb{R}^{M}
\]
$\textbf{V}$ is also known as \textbf{change of coordinates matrix} (also stated as any matrix whose columns form a \gls{basis}). If $\textbf{V}$ is orthogonal, then $\textbf{V}^{-1} = \textbf{V}^{T}$ and hence $\textbf{y} = \textbf{V}^{T}\textbf{x}$
\end{appendices}