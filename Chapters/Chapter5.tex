\externaldocument{chapter1}
\chapter{Conclusion} % Main chapter title

\label{Chapter5}
The model settings for the task of \textit{aesthetic tagging} have been analysed in this thesis. From the machine learning end, the performance can be further pushed by 
\begin{itemize}
\setlength\itemsep{0em}
\item Searching or testing other CNN models\cite{choi_transfer}
\item Working with the label space. For instance, there can be broad subsets and each tag can be belong to one or many of the subset. Then a separate model is trained for each subset 
\end{itemize}

\noindent However, it is seen that the performance gap is still huge to serve any real-time application arising from aesthetic auto-tagging. Hence, to develop an algorithm that could come close to human artist, just a dataset with clean tags is not sufficient but also proper understanding of mathematical modelling of musical discriminants (ref. \ref{discriminants}) is important.
\\
\\
\textbf{TODO:} more ideas, discussion   
