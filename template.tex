\documentclass{book}

\addtolength{\textwidth}{3cm}  \addtolength{\oddsidemargin}{-1.5cm}  \addtolength{\evensidemargin}{-1.5cm}
\addtolength{\textheight}{1cm} \addtolength{\topmargin}{-.3cm}

\usepackage{amssymb}
\usepackage{amsmath,bm}

\usepackage{booktabs}
\usepackage{aicescover}
\usepackage{hyperref}
\hypersetup{
    colorlinks=true, %set true if you want colored links
    linktoc=all,     %set to all if you want both sections and subsections linked
    linkcolor=blue,  %choose some color if you want links to stand out
}

\usepackage[
backend=bibtex8,
sorting=ynt,
maxbibnames=99
]{biblatex}
\addbibresource{bibliography.bib}

\linespread{1.1}
\pagestyle{plain}
\usepackage{graphicx}
\usepackage{caption}
\usepackage{subcaption}
\usepackage{enumitem}
\usepackage{placeins}
%\usepackage{algorithm}
\usepackage{algorithm}
\usepackage{algpseudocode}
\usepackage{glossaries}
\usepackage{appendix}
\usepackage{xr}
\usepackage{color}
\usepackage{longtable}
\DeclareMathOperator*{\argmin}{arg\,min} % Jan Hlavacek
\newcommand\norm[1]{\left\lVert#1\right\rVert}


\makeglossaries
\loadglsentries{glossary.tex}

\graphicspath{ {Figures/} }

\begin{document}

\aicescovertitle{Analyses of Convolutional Neural Networks for Automatic tagging of music tracks }
\aicescoverauthor{{\bf Master Thesis}\\Aravind Sankaran}
\aicescoversupervisor{Prof. Paolo Bientinesi \\ Prof. Marco Alunno}
\aicescoverexaminer{Prof. Paolo Bientinesi \\ Prof. Bastian Leibe}
\aicescoverpage


\section*{Abstract}

Describing music can be quite tricky and talking about music may simply require as much vocabulary as any technical subject. Musicians and composers usually discuss their work with jargon describing certain aesthetics of the song. As the amount of music recordings are constantly growing, finding a song that matches these aesthetic description is challenging. This work is an attempt to take a step towards developing algorithms that could tag music like an artist. The currently available state-of-art algorithms are trained and tested on datasets with tags that are socially biased. Moreover these datasets contain just short clips of songs, but an artist describes a song as a whole. Therefore, in this thesis, a repertoire of 900 songs with carefully labelled data describing the whole track is used and the aim is to find the model settings that would best approximate the audio features for such a dataset. The Mel-Frequency-Cepstral-Coefficients (MFCC) features are compared with features extracted by pre-trained Convolution Neural Networks (CNN) over mel-log power spectrogram. These features are extracted every 29.1s and approximated over time to a fixed size representation. The temporal approximation by sequence to one Long Term Short Memory (LSTM) Recurrent Neural Network is compared with approximation by Bag of Frames (BoF) approach and the weighted area under receiver operating characteristic curve (AUC) is reported. The experiments show that MFCC features summarized by LSTM outperforms pre-trained convolutions counterpart. It is also seen that LSTM perform better than Bag of Frames features for temporal approximation.     

\vspace{5cm}


\newpage

\section*{\bf Acknowledgments}
I would like to thank \textit{Prof. Paolo Bientinesi} (High performance and automatic comupting group, AICES, RWTH Aachen) for doing one of the most expensive work - \textit{Listen to almost thousand songs and tag them}. I also thank  \textit{Prof. Marco Alunno} (Professor of Composition and Theory at University EAFIT, Columbia) for his valuable association and advices.  
\tableofcontents

% Chapter 1
\externaldocument{chapter3}
\externaldocument{chapter4}
\chapter{Introduction} % Main chapter title

\label{Chapter1} % For referencing the chapter elsewhere, use \ref{Chapter1} 

%----------------------------------------------------------------------------------------

% Define some commands to keep the formatting separated from the content 
\newcommand{\keyword}[1]{\textbf{#1}}
\newcommand{\tabhead}[1]{\textbf{#1}}
\newcommand{\code}[1]{\texttt{#1}}
\newcommand{\file}[1]{\texttt{\bfseries#1}}
\newcommand{\option}[1]{\texttt{\itshape#1}}

%----------------------------------------------------------------------------------------

%\section{Welcome and Thank You}

Computers have been used to automate discovery and management of music in so many different ways. Automating the task of attaching a semantic meaning to a song is popularly known as \textit{music auto-tagging}. Automatic tagging algorithms have been used to build recommendation systems that allow listeners to discover songs that match their taste. It also enables online music stores to filter their target audience. But semantic description of a song is not straightforward and there is this gap between music audio and listener's description, both linguistically and emotionally, which we term as \textit{audio-semantic gap}. In this thesis, we try to develop an algorithm that is least affected by \textit{audio-semantic} noise. In section \ref{motivation}, the need for this dedicated research is explained by describing some shortcomings of the currently available solution approaches. In section \ref{structure}, a top to bottom overview of the contents of this research is presented.   

%----------------------------------------------------------------------------------------

\section{Motivation}
\label{motivation}
Describing music is a very general statement and it is difficult to assign an absolute value to it. This also makes analysing music and communicating it's artistic properties more challenging. Although great technical progress have been made to enable efficient retrieval and organization of audio content, the problem of music recommendation is complicated because of the sheer variety of genre,mood,acoustic scene, as well as social factors that affect listeners preference. The problems faced by current music recommendation systems are mentioned below,  

\subsection{Cold start problem with collaborative filtering methods}
When the usage data is available, one can use collaborative filtering to recommend the tracks on trending lists. That is, if a listener liked songs A and B and you liked A, you might also like B. Such algorithms have proven to be extremely efficient in finding similar songs and out-perform those algorithms that works by extracting acoustic cues from audio signal \cite{DC1}. However, in absence of such usage data, one resorts to content-based methods, where just the audio signal is used for generating recommendations. Thus collaborative filtering methods suffer from what is called a \textit{cold start problem}, making it less efficient for new and unpopular songs. 


\subsection{Problems with content based methods}
\label{problems}
Using information from audio content to overcome the cold-start problem resulted in \textit{content-based} recommendation methods. In such algorithms, a classifier is trained on some training data to learn acoustic cues. But a recommendation system can also be built without requiring  such acoustic labels by combining \textit{content based} and \textit{collaborative} techniques[combining content], where training is done from collaborative view point. However, this is not sufficient if a recommendation system has to designed for artists and composers who search for songs based on some properties of music itself. In such cases, the current \textit{content-based} methods fall short because of following training assumption,   

\subsubsection{Psychoacoustic assumptions}
Music descriptions are often affected by social factors. However, it is possible to measure what percentage of subjects classify a music to certain mood (say, happy, dull etc.) and present the popular description. The currently available large datasets are built on this assumption \cite{MSD}\cite{MTT}.  Training on such datasets may not help in developing algorithms that can actually extract the musical properties (also termed as \textit{aesthetics} in popular journals of music psychology [no account for taste..]). This psychoacoustic assumption stands as a barrier to discover songs based on aesthetics (which has applications in music therapy \cite{MusicTherapy}).

\subsubsection{Temporal summarization of audio content}
The current state of art music tagging algorithms\cite{choi_crnn}\cite{MultiScale} are established by training on datasets that contain just short music excerpts. While recommending, these algorithms classify each short section separately and merge tags across different sections. It is not clear if such section-wise mearging can approximate actual properties of the whole song. Hence there is a need to test algorithms that extract features approximating greater length of the song. 

\subsection{Need for adaptive glossary}
Current recommendation systems including the ones that use collaborative filtering, restrict the user with the choice of tags. Moreover, it is not guaranteed that all users will perceive all the tags in the same way. A recent study in idiographic music psychology have indicated that different people use different aesthetic criteria to make judgement about music\cite{NoAccountingForTaste}. Hence it would be interesting to study if these models \cite{choi_cnn}\cite{choi_crnn} trained on large datasets can be exploited for training on personal repertoire, which are usually small.  

%----------------------------------------------------------------------------------------

\section{Overview}
\label{overview}
In the following work, only content-based methods are considered for multi-label classification. That is to say that only raw signal is used as input for classification. This requires feature extraction from input audio, temporal summarization of features followed by classification. 

\subsection{Supervised training on repository with aesthetic tags }
The machine learning assumption is that if we show an algorithm enough examples, the correlation between human tags and predicted tags will become clear. Thus, a properly labelled training set is usually required to solve the task of automatic tagging. 

In this thesis, a repository labelled with aesthetic judgements of songs is used. An aesthetic judgement is a subjective evaluation of a piece of music as art based on an individual set of aesthetic properties. [from everyday .. ] An aesthetic judgement is assumed to rely more on ‘higher’ cognitive functions , domain relevant knowledge, and a fluid, individualized process that may change across time and context. We discuss how aesthetic judgements influence preference for a song. In a recent work in experimental music psychology, strong correlation have been found between the two. But it is also shown that such preferences also vary widely between cultural contexts. ( see ) This also justifies the need for adaptive glossaries in recommendation systems. 

\subsection{Convolutional neural network for feature extraction}
The input signal preprocessed to spectrogram is mapped to the feature space by convolving hierarchically with learnable filters. (see ch.2,1) Conceptual arguments have been made to demonstrate that deep processing models are powerful extensions of hand-crafted feature extraction methods\cite{Yann}. (see ch.3.1) It is also shown that deep layers learn to capture textures and patterns of continuous distribution on a spectrogram for music classification task. [explaining cNN for music class]. (see ch 3.2).  we discuss the ability of these CNN models to be fine tuned on a medium sized dataset (see ch ). Convolutions over log amplitude mel spectrogram and MFCC are studied (see ch)



\subsection{ Recurrent neural network for temporal summarization of features}
 The features extracted on every 29.1s time frame are then sent to sequence to one RNN. This leaves us with a feature of fixed dimension for audio of arbitrary length up to five minutes. (see ch ). Here we make an assumption that a listener can make an aesthetic judgement within 5 minutes. Conceptual comparisons of RNN with the state of art temporal feature pooling technique in [ ] have been discussed (see ) . Effectiveness of temporal summarization have been justified by comparing with the performance of section-wise merging of tags ( see )

\subsection{ classification }
The features are then mapped to the probability space of labels. Multilayer perceptron with binary cross entropy loss is used for training. End to end training is compared with two-stage method, separating  training of features and training of classifier ( see ch ). In the two-stage approach, MLP is compared with SVM classifier. ( see )



%----------------------------------------------------------------------------------------

\section{Outline of the report}
In chapter 2, the terminologies and mathematical formulations are elaborated. Advanced readers can skip this chapter. In chapter 3, a detailed overview of previous research, their shortcomings for the current problem along with justification for proposed models are discussed. In Chapter 4, details of the dataset, implementations and the experiment results of proposed models are discussed. In chapter 5, the results are analysed and the need for biologically motivated feature extraction techniques are discussed. 







% Chapter 2

\externaldocument{chapter3}
\externaldocument{chapter4}
\externaldocument{chapter5}
\externaldocument{chapter1}
\externaldocument{AppendixA}
\chapter{Formalisms} % Main chapter title

\label{Chapter2} % For referencing the chapter elsewhere, use \ref{Chapter2} 


%----------------------------------------------------------------------------------------

%\section{Welcome and Thank You}
%In this chapter, acoustical characteristics of music signal that enables general MIR tasks will be introduced.We will examine the Fourier Series representations of sound waves and see how they relate to harmonics and tonal color of instruments  
To map a music to it's semantic descriptions, the signal is first transformed to a lower dimensional space. The resulting transformation is called \textit{feature}. The classifier then takes these features as input and performs the classification task. The performance of the classifier can only be as good as the information encoded in the features. In this chapter, formalisms required for the analysis of feature extraction are introduced. The raw signal is first changed to a representation that can be mathematically analysed (see Sec. \ref{rep}). This representation is then transformed to a lower dimensional space by discarding information that does not contribute to discrimination (see Sec. \ref{dimension}). The resulting reduction is then approximated to a fixed size representation (see Sec. \ref{temporal}) which is then taken as input by a multi-label classifier (see Sec. \ref{classifier}). In section \ref{training}, training procedure that can optimize feature representations for supervised classification task is discussed. 
%----------------------------------------------------------------------------------------

\section{Representation of music signal}
\label{rep}
The observed signal is traditionally represented in the time domain. The time domain is a record of what happened to a parameter of the system versus time. Standard formats use amplitude versus time. The observed signal is then discretised by sampling and stored in digital format (see \ref{sampling}). This signal in the time domain is then changed to frequency domain (see \ref{time}). This is simply because our ear-brain combination is an excellent frequency domain analyser. Currently used music signal representations for general MIR tasks are explained in section \ref{stft}.


\subsection{Sampling of continuous-time signal}
\label{sampling}
The digital formats contain the discrete version of the signal obtained by sampling continuous-time signal. Sampling is performed by measuring the value of the continuous signal every $T$ seconds. This interval $T$ is called the sampling interval or the sampling period. The sampling frequency or sampling rate ($f_{s}$) is the number of samples obtained in one second (samples per second),  
\[
 f_{s} = \frac{1}{T}.
\]
The optimum sampling rate is given by Nyquist-Shannon sampling theorem which says, the sampling frequency $f_{s}$ should be at least twice the highest frequency contained in the signal. Given the human hearing range lies between 20Hz - 20KHz, most of the digital audio formats use a standard sampling frequency of 44.4Khz. Sampling rate determines the initial dimension of the raw signal. The signal may be further down sampled if higher sampling rate does not contribute to classification performance.

\subsection{Time-Frequency transformations}
\label{time}
The digital signal is represented in the time domain with amplitude values at each time $t$. This representation has to be changed to frequency domain. Mapping from time-domain to frequency-domain is looked up on as basis transformation. 

\subsubsection{Basis transformation from time to frequency domain}
\noindent The signal in the time domain $\textbf{a}$ is a set of ordered \textit{n}-tuples of real numbers \( (a_{1},a_{2}, ...,a_{N}) \in \mathbb{R}^N \) in the vector space \textit{V}, specifically \textit{Euclidean n-space}. That is to say, a discrete-time signal can be represented as a linear combination of Cartesian basis vectors. The coefficients in linear combination are then the co-ordinates of the corresponding basis system.  
\[
\textbf{a} = (a_{1},a_{2}, ...,a_{N}) = a_{1}\textbf{e}_{1} + a_{2}\textbf{e}_{2} + ... + a_{N}\textbf{e}_{N} = \displaystyle\sum_{t=1}^{N}a_{t}\textbf{e}_{t} = \mathbb{I}\textbf{a}
\]
Where $(a_{1},a_{2}, ...,a_{N})$ are the co-ordinates of Cartesian basis formed by basis vectors $\textbf{e}_{1} ... \textbf{e}_{N}$ with $t \in \{1,2,..N\}$. The unit vector $\textbf{e}_{t} \in \mathbb{R}^{N}$ has 1 in the $t^{th}$ index and 0 everywhere else.
\bigskip

\noindent To transform to frequency domain, we need to find a set of basis vectors $\bm{\phi}_{ \omega }$ that are functions of frequencies ($\omega$). Then the co-ordinates of this basis system $c_{ \omega }$ represents the signal in frequency domain. 
\begin{equation}
\label{exp_fourier}
\textbf{a} = \displaystyle\sum_{ \omega =1}^{M}c_{ \omega }\bm{\phi}_{ \omega } = \bm{\Phi}\textbf{c} \qquad \bm{\Phi} \in \mathbb{C}^{N \times M}, \textbf{c} \in \mathbb{C}^{M}
\end{equation}
If $\bm{\Phi}$ is known, then the transformed co-ordinates $\textbf{c} = (c_{1},c_{2},..,c_{M})$ can be computed as,
\[
\textbf{c} = \bm{\Phi}^{-1}\textbf{a}
\]
$\textbf{c}$ is the transformed representation. $\bm{\Phi}^{-1}$ is the operator that transforms the signal. 

\subsubsection{Exponential Fourier Series and Fourier Transform}
From the definition of \textit{exponential Fourier Series}, any function \textit{periodic} in $\{1,2,..,T\}$ can be expanded with series of complex exponentials\cite{allen}. These complex exponentials which are functions of harmonically related frequencies($k \omega$) form basis 
\[
\bm{\phi}_{k}(t) = \frac{1}{\sqrt{T}}e^{ik \omega t} \qquad t \in \{1,2,..T\}
\] 
It is difficult to assume periodicity for a generalized signal. Hence, the Fourier series can not be applied directly and Fourier Transform was developed. By Fourier transform, quantity of each frequency component $\omega$ in an arbitrary signal $\textbf{a}(t)$ can be computed by dividing by $e^{i \omega t}$. Application of Fourier transform to a discrete signal is called \textit{Discrete Fourier Transform}
\[
c_{\omega} =  \displaystyle\sum_{t=1}^{N}a_{t}e^{-i \omega t} \qquad t \in \{1,2,...,N\}
\] 
Thus the transformation operator is,
\[
\bm{\Phi}^{-1}[ \omega ] = e^{-i \omega \textbf{t}} \qquad \textbf{t} \in \mathbb{R}^{N}, \omega \in \{1,2,..,M\}
\]
Fast Fourier Transform(FFT) is an efficient implementation of Discrete Fourier Transform(DFT) which exploits the symmetry of $sines$ and $cosines$ in the complex exponential. While DFT requires $O(N^2)$ operations, FFT requires only $O(NlogN)$ \cite{allen}. 


\subsection{STFT, Mel-Spectrogram}
\label{stft}
It is useful to perform FFT locally over short segments. This is because we are more interested in the evolution of frequency content rather than the frequency content of the entire signal. Hence, the full length signal is divided into short segments, and FFT is computed separately for each segment. This is known as \textbf{Short Time Fourier Transform (STFT)}. One common problem with STFT is the \textit{spectral leakage}, which is addressed by modifying the original signal with some window function. The most common window function is the \textbf{Hamming Window} defined as,
\begin{equation}
h[n] = 0.54 - 0.46cos(\frac{2 \pi n}{N-1})
\end{equation}
where $n \in \{1,2,..,F\}$ and $F$ is the size of window function. The signal approaches zero near $n=1$ and $n=N$, but reaches peak near $n=N/2$ \cite{specLeak}. To overcome the information loss at the ends of the window, signal is divided into segments that are partly \textit{overlapping} with each other. The distance between the start of two adjacent segments is called \textit{hop-length}. Figure \ref{fig:stftPipe} shows the extraction of spectral frames of a spectrogram. Thus, the parameters of STFT include 
\begin{itemize}
    \setlength\itemsep{0em}
    \item Choice of window function
    \item Size of each segment in $\textbf{a}$ ($F$)
    \item Hop length or stride ($s$) of the transformation operator
    \item Size of frequency dimension $M$ (also known as FFT size)
\end{itemize}

\begin{figure}[h]
       \begin{subfigure}[b]{0.6\textwidth}
        \includegraphics[width=\textwidth]{stft_pipe}
        \caption{Windowing is applied on overlapping segments\\ followed by FFT }
        \label{fig:stftPipe}
       \end{subfigure}
	    \begin{subfigure}[b]{0.4\textwidth}
        \includegraphics[width=\textwidth]{stft}
        \caption{
        Application of Hamming Window on \\a segment of input signal
        }
        \label{fig:HammingWindow}
       \end{subfigure}
       \caption{\cite{spec_dia} (a) Shows STFT Pipeline. (b) Shows the application of \\Window function}\label{fig:STFT}
\end{figure}
\FloatBarrier

\subsubsection{STFT as Convolution}
\noindent The strided operation over the signal $\textbf{a}$ is mathematically represented as a \textbf{convolution}. The resulting spectrogram has $P$ frames. The discrete STFT (\textit{slow}) for $p^{th}$ frame of signal $\textbf{a}$ is obtained as,
\begin{equation}
\label{stfteq}
\textbf{C}[p, \omega ] = \displaystyle\sum_{n=p.s}^{p.s + F}\textbf{a}[n] \Big( \textbf{h}[n-p.s] . e^{-i \omega (n-p.s)} \Big)
\end{equation}
Where:
\begin{itemize}[label=]
    \setlength\itemsep{0em}
    \item $P$: is the number of spectral frames; $p \in [1,2...,P]$ 
    \item $M$: is the dimension of discrete frequency space ; $\omega \in \mathbb{R}^{M}$
    \item $F$: is the frame length 
    \item $s$: is stride (or) hop-length to the next segment
    \item $\textbf{a} \in  \mathbb{R}^{N}$ ; $n \in [1,2,...,N]$
    \item $\textbf{h} \in  \mathbb{R}^{F}$
    \item $\omega \in  \mathbb{R}^{M}$
    \item $\textbf{C} \in \mathbb{C}^{M \times P}$
\end{itemize}
\noindent Equation \ref{stfteq} can be seen as a \textbf{discrete convolution} over the signal \textbf{a} with the operator $\textbf{W}_{STFT}$ which has finite support over the set $\{1..,F\}$ with stride $s$
\[
\textbf{C} = \textbf{a} \star \big( \textbf{h} \bm{\Phi}^{-1} \big)
\]
\begin{equation}
\label{eq:stft}
\textbf{C} = \textbf{a} \star {\textbf{W}^{(s)}}_{STFT}
\end{equation}
where $\textbf{W}_{STFT} = \textbf{h} \bm{\Phi}^{-1}$ is the STFT operator that transforms the signal $\textbf{a}$ from time to frequency domain. 
\bigskip

\subsubsection{Power spectrogram}
\noindent It is important to note that the coefficient matrix $\textbf{C}$ may be complex valued. To obtain useful metrics, we need to extract some physical quantity from the coefficients. This is where \textbf{Parseval's theorem} is used, which relates time and frequency domain components in DFT as follows \cite{allen} :
\begin{equation}
{\|\textbf{c}\|}^2 \propto {\|\textbf{a}\|}^2
\end{equation}
If \textbf{a} represents amplitude in the time-domain, then we know that the energy of a wave is related to it's amplitude as,
\begin{equation}
Energy \propto amplitude^2
\end{equation}
Thus, it can be inferred that \textbf{square} of the Fourier coefficients is proportional to the energy distributed in the corresponding frequencies. This spectrogram with squared coefficients is called the \textbf{Power Spectrum (P)}. It is often motivating to use this representation because \textit{loudness} is proportional to \textit{energy}.
\begin{equation}
\label{energy}
\textbf{P} = \textbf{C} \odot \textbf{C}
\end{equation}

\noindent The frequencies in the considered range are  grouped into bins. It is useful to do so, due to the aliasing effect of human auditory system. This is motivated by the human cochlea (an organ in the ear) which vibrates at different spots depending on the frequency of the incoming sounds.
  
\subsubsection{Mel Power Spectrogram}
\label{mel}

The \textit{mel-scale} was developed to express measured frequency in terms of psychological metrics (i.e perceived pitch). The mel-scale was developed
by experimenting with the human ears interpretation of a pitch. The pitch is linearly perceived in the frequency range 0-1000 Hz. Above
1000 Hz, the scale becomes logarithmic. There are several formulae to convert Hertz to mel. The following formula is used in this thesis\cite{speech}
\begin{equation}
\omega_{m} = 2595log_{10}\bigg(1+\frac{ \omega }{700}\bigg)
\end{equation}
where $\omega$ is the frequency in Hertz. In a mel spectrogram, the frequencies and converted to mels and then grouped into mel-spaced bins. This is done by multiplying the spectrum with \textbf{mel filter bank ($\textbf{W}_{MEL}$)}. For details about computation of mel-filter banks, refer \cite{mel}. Each filter bank is centered at a specific frequency. Hence, to compute R mel bins, we need R mel-filter banks. The resulting mel power spectrogram ($\textbf{X}$) is
\begin{equation}
\label{mel_conv}
\textbf{X} = \textbf{W}_{MEL}\textbf{P}
\end{equation}
\[
 \textbf{W}_{MEL} \in  \mathbb{R}^{R \times M}, \textbf{P} \in \mathbb{R}^{M \times P}, \textbf{X} \in \mathbb{R}^{R \times P}
\]

\noindent The above Matrix-Matrix multiplication can be represented as a convolution with window size and stride equal to $M$. We can re-write equation \ref{mel_conv} as, 
\begin{equation}
\label{mel_conv_flat}
\textbf{X}[p,\omega_{m}] = \displaystyle\sum_{k=p.M}^{p.M + K}\textbf{p}[k]\textbf{W}_{MEL}(k-p.M)
\end{equation}
Where:
\begin{itemize}[label=]
    \setlength\itemsep{0em}
    \item $\textbf{p}[k]$ = $\textbf{P}[i,j]$ ; $i = floor(\frac{k}{M})$ ; $j = k-floor(\frac{Mk}{M-1})$
    \item $\omega = k-p.M \in \mathbb{R}^{M}$
    \item $\textbf{X} \in \mathbb{R}^{M \times P}$
    \item $\textbf{p} \in \mathbb{R}^{M.P}$
    \item $\omega_{m} \in  \mathbb{R}^{R}$
\end{itemize}

\noindent Hence, we can represent mel-power spectrogram as \textbf{M-strided convolution} over \textit{flattened} $\textbf{P}$ with mel filters $\textbf{W}_{MEL}$ (i.e, the frequency axis of $\textbf{P}$ is contracted with each mel-filter) 
\begin{equation}
\label{eq:mel}
\boxed
{
  \textbf{X} = \textbf{P} \star \textbf{W}_{MEL}
}
\end{equation}  
Thus the extraction of mel-power spectrogram can be summarized in the following algorithm 
\begin{algorithm}
  \caption{$\textbf{X}$ = $R_{(MEL)}$($\textbf{a}$)}\label{alg:mel}
  \begin{algorithmic}[1]
    \Statex \textbf{Input :} $\textbf{a} \in \mathbb{R}^{N}$
    \Statex \textbf{Output :} $\textbf{X} \in \mathbb{R}^{R \times P}$
	\State $\textbf{C} = \textbf{a} \star \textbf{W}_{STFT}$
	\State $\textbf{P} = \textbf{C} \odot \textbf{C}$
	\State $\textbf{X} = \textbf{P} \star \textbf{W}_{MEL}$
  \end{algorithmic}
\end{algorithm}
\FloatBarrier

\section{Dimensionality Reduction}
\label{dimension}
The objective of dimensionality reduction is to retain only the information that contribute to discrimination and discard the rest. This is done because the \textit{representation} ($\textbf{X}$) can be large for longer audio tracks (because number of frames $P$ depends on length of the audio). In this thesis, only \textit{mel-spectrogram representation} is considered. The dimensionality reduction of input signal $\textbf{a}$ is generalized as follows,
\[
   \textbf{X} = R(\textbf{a}) \qquad R : \mathbb{R}^{N} \rightarrow \mathbb{R}^{R \times P}
\]
\begin{equation}
\label{dim_red_abstract}
   \textbf{Y} = D(\textbf{X}) \qquad D : \mathbb{R}^{R \times P} \rightarrow \mathbb{R}^{T \times W} 
\end{equation}
 
\noindent The computation of mel-spectrogram defined in the previous section can be a part of the function $R$. The dimensionality reduction is defined by function $D$. The output of reduction $\textbf{Y} \in \mathbb{R}^{T \times W}$ will be the reduced representation ($T < R$ or $W < P$). Depending on how the function $D$ is defined, the following three methods will be discussed
\begin{itemize}
  \item \textbf{Principal Component Analysis} (PCA) : Reduction by \textit{unsupervised learning}.
  \item \textbf{Mel-Frequency Cepstrum Coefficients} (MFCC) : Reduction by domain engineering.
  \item \textbf{Convolution Neural Networks} (CNN) : Reduction by \textit{supervised learning}
\end{itemize} 
\begin{figure}[h] 
\centering
\includegraphics[width=0.7\textwidth]{dim_red}
\caption{Dimensionality Reduction Pipeline}
 \label{fig:Dimensionality Reduction}
 \end{figure}
\FloatBarrier
\bigskip

\subsection{Domain engineering Vs Unsupervised learning Vs Supervised learning}
\textit{Reduction by Domain engineering :} When the operators performing reduction are computed using the domain specific properties of the data, then the resulting reduction is said to be \textit{engineered}. Coming up with such operators is usually time-consuming and requires expert knowledge.\\
\\
\textit{Reduction by Unsupervised learning :} When the operators performing reduction are computed by exploiting the representation \textit{structure} of the data, then the resulting reductions are said to be \textit{learned} from the data. When this learning problem \textit{does not} require any labelled data, then the reduction is said to be \textit{unsupervised}.\\
\\
\textit{Reduction by Supervised learning :} When the operators performing reduction computed by exploiting the information from labelled data, then the resulting reduction is said to be obtained by \textit{supervised learning} 

\subsection{Principal Component Analysis (PCA)}
\label{pca}

The representation $\textbf{X}$ is changed to a \textit{basis} that are functions of variance between the frequency components. This is done by computing the co-variance matrix $\bm{\Sigma}$ from the data and performing \textit{orthogonal decomposition} to compute it's basis. The co-ordinates of the resulting basis system are called \textit{principal components}. The key idea for reduction is to discard the information corresponding to \textit{low variance} basis. The computation of PCA reduction operator $\textbf{W}_{PCA}$ is elaborated below,  

\begin{enumerate}[label=(\alph*)]
\item Usually, large samples (say $S$ samples) of representations from the dataset ($ \textbf{S} = [\textbf{X}_{1}, \textbf{X}_{2}, ..., \textbf{X}_{S}])$ are used to compute the co-variance matrix. The columns of $\textbf{S}$ are centred by their mean and the covariance matrix is computed as,
\[
   \bm{\Sigma} = \textbf{E}[(\textbf{S} - \textbf{E}[\textbf{S}])(\textbf{S} - \textbf{E}[\textbf{S}])^{T}] = \frac{1}{\displaystyle\sum_{s}{P_{s}}}\textbf{\^S}\textbf{\^S}^{T} \quad \in \mathbb{S}^{R \times R}
\]
The covariance matrix $\bm{\Sigma}$ is symmetric positive definite and hence belongs to space of symmetric operators $\mathbb{S}$.
\item The eigen values and eigen vectors of $\bm{\Sigma}$ are computed. At this point, we use the Orthogonal Eigenvector Decomposition Theorem and infer that eigen vectors of symmetric matrix ($\bm{\Sigma}$) form an orthogonal basis in $\mathbb{R}^{R}$. 
\[
\bm{\Sigma} = \textbf{V}\bm{\Lambda}\textbf{V}^{T} \qquad \textbf{V} \in \mathbb{O}^{R \times R}, \quad \bm{\Lambda} \in \mathbb{D}^{R \times R}
\]
The columns of matrix $\textbf{V}$ form the basis. Since the columns are orthogonal to each other, $\textbf{V}$ belongs to a space of orthogonal operators $\mathbb{O}$. $\bm{\Lambda}$ is a diagonal matrix of eigen values.

\item The eigen values represent the magnitude of variance for each frequency dimension.  Hence, eigenvectors corresponding to large eigen values gives the coordinates corresponding to greater variance. So eigen vectors corresponding to top $T$ eigen values are retained, while ignoring coordinates of lower variance. The resulting change of coordinates matrix is then $\textbf{\^V} \in \mathbb{O}^{R \times T}$
   
\item Since $\textbf{\^V}$ is orthogonal, $\textbf{\^{V}}^{-1} = \textbf{\^{V}}^{T}$, and the resulting reduction for \textit{each sample} can computed as $\textbf{Y} = \textbf{\^{V}}^{T}\textbf{X}$, where $\textbf{X} \in \mathbb{R}^{R \times P}$, $\textbf{Y} \in \mathbb{R}^{T \times P}$ and $T < R$. Thus the reduction operator is
\[
\textbf{W}_{PCA} = \textbf{\^V}^{T}
\]

\end{enumerate}   

\begin{algorithm}
  \caption{$\textbf{W}_{PCA}$ = PCA($\textbf{X}_{1}, \textbf{X}_{2},..,\textbf{X}_{S}$)}\label{alg:pca}
  \begin{algorithmic}[1]
    \Statex \textbf{Input :} $\textbf{S} = [\textbf{X}_{1},..\textbf{X}_{S}] \qquad \textbf{X}_{s} \in \mathbb{R}^{R \times P_{s}}, \textbf{S} \in \mathbb{R}^{R \times Q},\qquad  Q = \displaystyle\sum_{s}{P_{s}}$
    \Statex \textbf{Output :} $\textbf{W}_{PCA} \in \mathbb{R}^{T \times R}$
      \State $\bm{\Sigma} = \frac{1}{Q}\textbf{S}\textbf{S}^{T}$ \Comment{$\bm{\Sigma} \in \mathbb{S}^{R \times R}$}
      \State $\textbf{V}^{T} \bm{\Lambda} \textbf{V} = \bm{\Sigma}$ \Comment{$\bm{\Lambda} \in \mathbb{D}^{R \times R}$ , $\textbf{V} \in \mathbb{O}^{R \times R}$}
      \State $\textbf{V} \leftarrow \textbf{V}[:][:T]$ \Comment{$\textbf{V} \in \mathbb{O}^{R \times T}$}
      \State $\textbf{W}_{PCA} = \textbf{V}^{T}$
  \end{algorithmic}
\end{algorithm}
\FloatBarrier

\noindent Since the operator $\textbf{W}_{PCA}$ is computed from the data without requiring labelling, this method falls under \textit{unsupervised learning}. The columns of representation are first centred before applying the PCA reduction operation. An illustration of PCA reduction function $D_{(PCA)}$ is shown below,

\begin{algorithm}
  \caption{$\textbf{Y}$ = $D_{(PCA)}$($\textbf{X}$)}\label{alg:dpca}
  \begin{algorithmic}[1]
    \Statex \textbf{Input :} $\textbf{X} \in \mathbb{R}^{R \times P}$
    \Statex \textbf{Output :} $\textbf{Y} \in \mathbb{R}^{T \times P}$
	\State $\textbf{\^X} = \textbf{X} - \textbf{E}[\textbf{X}]$
	\State $\textbf{Y} = \textbf{W}_{PCA}\textbf{\^X}$
  \end{algorithmic}
\end{algorithm}
\FloatBarrier

\noindent Sometimes, to make the resulting reduction $\textbf{Y}$ uncorrelated (the resulting transformation should have identity co-variance matrix), the corresponding dimensions are divided by their eigen values. This is because eigen values are proportional to the magnitude of variance in each frequency direction. This operation is known as \textbf{PCA Whitening} and the reduction operator is,
\[
\textbf{W}_{PCAW} = \bm{\Lambda}^{-1}\textbf{\^V}^{T}
\]    
The reduction function $D_{(PCAW)}$ is same as algorithm \ref{alg:dpca}, except that the operator $\textbf{W}_{PCAW}$ is used instead of $\textbf{W}_{PCA}$
\bigskip

\subsection{Mel-frequency cepstrum coefficients (MFCC)}
\label{mfcc}
It has been studied that the basis functions resulting from co-variance matrix of log-mel spectrogram representation are similar to cosine transform on log-mel spectrogram\cite{mfcc_pca}. Therefore, instead of explicitly computing the basis functions from the data, one can simple use cosine basis. The co-ordinates of corresponding basis system are known as \textit{Mel-Frequency Cepstrum Coefficients}. The co-ordinates of high frequency cosine functions are discarded because they correspond to low-variance information. 
\bigskip

\noindent This reduction is engineered only for log mel spectrogram representation. $\textbf{W}_{MFCC}$ is the reduction operator.

\begin{algorithm}
  \caption{$\textbf{Y}$ = $MFCC$($\textbf{a}$) }\label{MFCC}
  \begin{algorithmic}[1]
    \Statex \textbf{Input :} $\textbf{a} \in \mathbb{R}^{N}$
    \Statex \textbf{Output :} $\textbf{Y} \in \mathbb{R}^{T \times P}$
    \State $\textbf{X} = R_{(MEL)} \big( \textbf{a} \big) $ \Comment{$\textbf{X} \in \mathbb{R}^{R \times P} $} 
    \State $\textbf{X} \leftarrow ln(\textbf{X})$
    \For{$ \omega \in \{1,..,T\}$}
    \State $\textbf{W}_{MFCC}[ \omega ] \leftarrow \textbf{cos}( \omega \textbf{t})$  \Comment{$\textbf{W}_{MFCC} \in \mathbb{R}^{R \times T}, \textbf{t} \in \mathbb{R}^{R}$}
    \EndFor
    \State $\textbf{Y} \leftarrow \textbf{W}_{MFCC}\textbf{X}$
  \end{algorithmic}
\end{algorithm}
 

\subsection{Convolution neural network}
\label{stacked}
Transformation of input $\textbf{a}$ by shifted contractions of an operator $\textbf{w}$ is termed as \textit{discrete convolution} and can be mathematically represented as one dimensional convolution operation, 
\[
	\textbf{y} = \textbf{a} \star \textbf{w}^{(s)}
\]   
The operator $\textbf{w}$ is known as a \textit{filter function} and the length of shift $s$ is known as \textit{stride}. Usually $\textbf{a}$ is convolved with multiple filter functions. For $K$ \textit{filters},
\[
	\textbf{Y}[k] = \textbf{a} \star {\textbf{w}_{k}}^{(s)} = \textbf{a} \star \textbf{W}^{(s)} \qquad k \in {0,1..K-1}
\] 
The index based notations for this operation are shown in appendix \ref{1dconv}. If the filters $\textbf{w}_{k}$ are \textit{defined}, then computation of $\textbf{Y}$ is straight forward. All the operations explained in the previous section are forward convolutions with defined filters,
\begin{itemize}
\setlength\itemsep{0em}

\item \textbf{STFT} in equation \ref{eq:stft}, where the filter functions are complex negative exponentials.
\item Transformation to \textbf{Mel} frequency scale in equation \ref{eq:mel}, where a set of mel-frequency centred filters are used.
\item \textbf{Principal components} and \textbf{MFCC} can be realized as a convolution of the input with transformation matrix $\textbf{V}^{T}$ defined in section \ref{pca} (Recall that matrix multiplication can be represented as a convolution of stride equal to column length. An illustration was shown in section \ref{mel})  

\end{itemize}
However, it is not clear if such defined filters really encodes the information in $\textbf{Y}$ relevant for certain context-based classification task. But, when a set of task-specific observations are available, it is possible to solve for the filters $\textbf{w}_{k}$, so that the resulting transformation $\textbf{Y}$ is optimal for the considered task. This is done using iterative optimization techniques, starting by random initializing of $\textbf{w}_{k}$ and updating it's values after every iteration. Computational models that solves by \textbf{first-order gradient descent} methods (a class of optimization techniques) are represented as first order \textbf{artificial neural networks}. The filters $\textbf{w}_{k}$ which we are attempting to solve are also termed as \textit{parameters} or \textit{weights} of the network. The iterative steps involving finding these \textit{weights} is termed as \textit{training} the neural network (Details of training neural network are explained in section \ref{training}). Since the transformation operation by the \textit{filters} are represented as convolution over the input, the neural network is termed as \textbf{convolution neural network}.   

\subsubsection{Approximating MFCC with CNN}
\textit{Supervised} feature learning has an advantage over \textit{unsupervised} feature learning when we wish to find filters $\textbf{w}_{k}$ optimal for context based classification tasks. Feature learning with CNN fall under \textit{supervised} feature learning category.
To show that CNN can do atleast as good as MFCCs, an illustration is discussed by approximating MFCC with CNN. Setting up a CNN architecture requires defining the \textit{hyper parameters} like number of filters, filter dimensions and stride of the filter. The domain knowledge of MFCC computation is used to set these hyper parameters.
\bigskip  

\noindent To compute MFCC features, the input signal $\textbf{a}$ is convolved with complex negative exponentials ($\textbf{W}_{STFT}$). After element-wise squaring operation, the resulting transformation is convolved with mel-filters ($\textbf{W}_{MEL}$). Logarithm of the resulting transformation is then convolved with cosine filters ($\textbf{W}_{MFCC}$). The motivation and description of these so called \textit{engineered} filters were described in section \ref{time} and \ref{dimension}. 

\begin{algorithm}
  \caption{$\textbf{Y}$ = MFCC($\textbf{a}$) }\label{MFCC_engineer}
  \begin{algorithmic}[1]
    \Statex \textbf{Input :} $\textbf{a} \in \mathbb{R}^{N}$
    \Statex \textbf{Output :} $\textbf{Y} \in \mathbb{R}^{T \times Q}$ 
    \State $\textbf{C} = \textbf{a} \star {\textbf{W}_{STFT}}^{(s)}$ \Comment{$ \textbf{W}_{STFT} \in \mathbb{R}^{M \times F}, \textbf{C} \in \mathbb{C}^{M \times Q}$}
    \State $\textbf{C} \leftarrow \textbf{C} \odot \textbf{C}$ 
    \State $\textbf{X} = \textbf{C} \star {\textbf{W}_{MEL}}^{(M)}$ \Comment{$ \textbf{W}_{MEL} \in \mathbb{R}^{R \times M}, \textbf{X} \in \mathbb{R}^{R \times Q}$}
    \State $\textbf{X} \leftarrow ln(\textbf{X})$
    \State $\textbf{Y} = \textbf{X} \star {\textbf{W}_{MFCC}}^{(R)}$ \Comment{$\textbf{W}_{MFCC} \in \mathbb{R}^{T \times R}$}
  \end{algorithmic}
\end{algorithm}
\FloatBarrier
\noindent An equivalent of MFCC computation can be realized with three layers of convolution by replacing the engineered filters by learnable filters and setting the following hyper parameters :
\begin{itemize}
\setlength\itemsep{0em}
\item $\textbf{W}_{L1}$ : $M$ filters (representing discrete frequencies) of size $F$ (STFT window size) and stride $s$ (STFT hop length)
\item $\textbf{W}_{L2}$ : $R$ filters (for mel-frequencies) of size $M$ and stride $M$.
\item $\textbf{W}_{L3}$ : $T$ filters (for mel coefficients) of size $R$ and stride $R$.   
\end{itemize}
The non-linearity in between each layer is needed. Otherwise, the filters accross layers can be combined into a representation for single layer. Setting $\bm{\Phi}_{1}$ as element-wise squaring operation and $\bm{\Phi}_{2}$ as logarithm should result in a CNN architecture that \textit{may} realize MFCCs. That is, the solution will converge to the defined filters ($\textbf{W}_{STFT}, \textbf{W}_{MEL}, \textbf{W}_{MFCC}$) if they are optimal for the task considered. Otherwise, one could expect either richer representation or sub-optimal representation because of local convergence.  
\begin{algorithm}
  \caption{$\textbf{Y}$ = CNN($\textbf{a}$) }\label{MFCC_learn}
  \begin{algorithmic}[1]
    \Statex \textbf{Input :} $\textbf{a} \in \mathbb{R}^{N}$
    \Statex \textbf{Output :} $\textbf{Y} \in \mathbb{R}^{T \times Q}$ 
    \State $\textbf{C} = \bm{\Phi}_{1} (\textbf{a} \star {\textbf{W}_{L1}}^{(s)})$ \Comment{$ \textbf{W}_{L1} \in \mathbb{R}^{M \times F}, \textbf{C} \in \mathbb{C}^{M \times Q}$}
    \State $\textbf{X} = \bm{\Phi}_{2} (\textbf{C} \star {\textbf{W}_{L2}}^{(M)})$ \Comment{$ \textbf{W}_{L2} \in \mathbb{R}^{R \times M}, \textbf{X} \in \mathbb{R}^{R \times Q}$}
    \State $\textbf{Y} = \bm{\Phi}_{3} (\textbf{X} \star {\textbf{W}_{L3}}^{(R)})$ \Comment{$\textbf{W}_{L3} \in \mathbb{R}^{T \times R}$}
  \end{algorithmic}
\end{algorithm}
\FloatBarrier
\noindent It is possible to with-hold the engineered filters at earlier levels and just introduce learning at later stages. For instance, it is sometimes optimal to compute the STFT and mel-filters and just perform convolutions over the mel-spectrogram\cite{EndToEnd}\cite{choi_cnn}. Sometimes it is also useful to generalize the convolution as a 2 dimensional operation\cite{MusicMotive}.(see Chapter 3, Sec. \ref{convolution}). 2D convolution operation is shown in appendix \ref{2dconv}. 

\subsubsection{CNN as a general purpose feature extractor}  
\label{general}
In music information retrieval, several task-specific features have been engineered. The MFCC features along with it's derivatives (the derivative is useful to encode the temporal evolution) are often used for genre and mood recognition tasks. Instead of convolving with mel-filters on STFT representation, one might opt for filters that would result in chroma-gram (combines frequencies by exploiting the periodicity of pitches) or tempo-gram (encodes change of frequencies over time). Features derived from chroma-gram find application for automatic mixing, chord recognition tasks. Features derived from tempo-gram find applications for tasks like on-set detection, tempo-estimation etc.. But more often, combination of features are used to boost classifier performance. For all these features that can be realized in terms of hierarchy of convolution operations, it is possible to find mathematical equivalence in convolution neural network.     

\begin{figure}[h] 
\centering
\includegraphics[width=0.95\textwidth]{dnn_motivation}
\caption{General purpose feature extractor}
 \label{fig:deep learning}
 \end{figure}
\FloatBarrier
\bigskip

\section{Temporal Approximation}
\label{temporal}
The size of resulting features computed through reduction operations discussed in the previous section depends on length of the audio. But classifiers like \textit{support vector machines} and \textit{multi layer perceptron} requires features of fixed size to compute the parameters while training. Therefore, frame-wise features over time are approximated to a fixed size feature representation. Temporal approximation $T$ of the reduced representation $\textbf{Y}$ is  
\[
\textbf{f} = T(\textbf{Y}) \qquad \textbf{f} \in \mathbb{R}^{Z}, \textbf{Y} \in \mathbb{R}^{T \times W}
\]
$W$ depends on length of the audio. $\textbf{f}$ is usually the final fixed size feature that is given as input to the classifier.  
\bigskip

\noindent The transformation by approximation function $T$ is done by using one of the methods,
\begin{itemize}
\setlength\itemsep{0em}
\item \textit{unsupervised} methods involving \textit{clustering techniques} (Bag Of Frames, Gaussian mixture models)
\item \textit{supervised} methods with \textit{recurrent neural network}. 
\end{itemize} 
 
\subsection{Bag Of Frames}
\label{clustering}
In Bag of Frames(BoF) model\cite{BoF}, the \textit{frequency} of each reduced frame is used as a feature for training a classifier. To reduce to a fixed size representation (of size $Z$), $Z$ cluster centres are computed from the data. Each frame-wise feature is assigned to it's nearest cluster. The number of assignments to each cluster now becomes the feature $\textbf{f}$. 
\begin{algorithm}
  \caption{$\textbf{f}$ = BagOfFrames($\textbf{Y}$) }\label{bof}
  \begin{algorithmic}[1]
    \Statex \textbf{Input :} $\textbf{Y} \in \mathbb{R}^{T \times W}$
    \Statex \textbf{Output :} $\textbf{f} \in \mathbb{R}^{Z}$
    \State $\textbf{f} = 0$
    \For{$ i \in \{0,1,..,W-1\}$}
    \State $j = \argmin\limits_{j} \norm{\textbf{Y}[:,i] - \bm{\mu}_{j}}^{2}$ \Comment{$\bm{\mu}_{j} \in \mathbb{R}^{T}, j \in \{0,1..Z-1\}$}
    \State $\textbf{f}[j] \leftarrow \textbf{f}[j] + 1$
    \EndFor
  \end{algorithmic}
\end{algorithm}
\FloatBarrier
\noindent The cluster centres are computed by clustering algorithms likes gaussian mixture models or K-Means clustering. Computation of $\bm{\mu}_{j}$ by K-Means is illustrated in algorithm \ref{alg:kmeans}. The cluster centres are computed for the entire dataset. Each of the $N$ sample track could have $W$ frame-wise features. To compute these centres, $\bm{\mu}_{j}$ are first randomly initialized. Then each frame level feature is assigned to the nearest centre. After the assignment of all frame features, the cluster centres $\bm{\mu}_{j}$ are re-computed. The frame features are then re-assigned to these new centres. This re-assignment and re-computation of $\bm{\mu}_{j}$ is iterated until convergence.  
\begin{algorithm}
  \caption{K-MEANS($\textbf{Y}_{0}, \textbf{Y}_{2},..., \textbf{Y}_{N-1}$) }\label{alg:kmeans}
  \begin{algorithmic}[1]
    \Statex \textbf{Input :} $\textbf{Y}_{n} \in \mathbb{R}^{T \times W}, n \in \{0,1,...,N-1\}$
    \Statex \textbf{Output :} $\bm{\mu}_{j} \in \mathbb{R}^{T}, j \in \{0,1,..,Z-1\}$
    \State Randomly initialize $\bm{\mu}_{j}$
    \While{$\bm{\mu}_{j}$ have not converged}
    \For{$n \in \{0,1,..,N-1$}
      \For{$ i \in \{0,1,..,W-1\}$}
         \State $j = \argmin\limits_{j} \norm{\textbf{Y}_{n}[:,i] - \bm{\mu}_{j}}^{2}$
          \State Assign $\textbf{Y}_{n}[:,i]$ to cluster $j$
      \EndFor
    \EndFor
   \State Recompute cluster means $\bm{\mu}_{j}$
   \EndWhile
  \end{algorithmic}
\end{algorithm}
\FloatBarrier

\subsection{Recurrent Neural Networks}
\label{rnn}
The idea behind RNN is to learn a feature representation by sequentially combining the input into an internal state $\textbf{h}$. The resulting feature ($\textbf{f}$) is a projection from this internal state.
\[ 
\textbf{f} = \textbf{W}\textbf{h}_{W} \qquad \textbf{W} \in \mathbb{R}^{Z \times K}, \textbf{h}_{W} \in \mathbb{R}^{K}
\]
\[
\textbf{h}_{W} = \bm{\Theta}(\textbf{Y}[:,W], \textbf{h}_{W-1}) \qquad \textbf{Y} \in \mathbb{R}^{T \times W}
\]
$\textbf{h}_{W}$ is the internal state after combining $W$ columns of $\textbf{Y}$. $\bm{\Theta}$ is the internal state function which sequentially combines the input. The operator $\textbf{W}$ and other operators resulting from the function $\bm{\Theta}$ are solved for optimality using a labelled dataset. These operators can be computed by training an RNN (see Section \ref{training}). The function $\bm{\Theta}$ should sufficiently hold the information from beginning to end of the sequence. RNNs face challenge in remembering the information in the earlier part of the sequence. Because of this, different flavours of RNN have been developed to hold the sequence information longer and to battle the vanishing gradient problem while training neural network (see Section \ref{training}). Depending on how $\bm{\Theta}$ is defined, at least two RNN architectures are popular in the literature: Long-Short Term Memory (LSTM) RNN and Gated Recurrent Unit(GRU) RNN. In this thesis, only LSTMs\cite{LSTM} are considered.   

\subsubsection{Long-Short Term Memory RNN}
The LSTM internal state is contained by three gates that control the information flow. This is done by multiplying the output of each sequence $\textbf{o}_{w}$ with the corresponding cell state function $\textbf{c}_{w}$.
\[
\textbf{h}_{w} = \textbf{o}_{w} \odot \sigma_{h}(\textbf{c}_{w}) \qquad w \in \{1,2,...,W\}
\] 
$\sigma_{h}$ is hyperbolic tangent function which projects the values between -1 and 1. The output gate $\textbf{o}_{w}$ combines the $w^{th}$ vector in sequence with the previous internal state ($\textbf{h}_{w-1}$) and projects the result between 0 and 1 with sigmoid activation ($\sigma$). This indicates the contribution of current sequence $w$ to the cell state. 
\[
\textbf{o}_{w} = \sigma(\textbf{W}_{o}\textbf{Y}[:,w] + \textbf{U}_{o}\textbf{h}_{w-1})
\]
The cell state $\textbf{c}_{w}$ acts as a conveyor belt where the information can either flow unchanged or get modified with update ($\textbf{i}_{w}$) and forget functions ($\textbf{g}_{w}$).
\[
\textbf{c}_{w} = \textbf{g}_{w} \odot \textbf{c}_{w-1} + \textbf{i}_{w} \odot \sigma_{h}(\textbf{W}_{c}\textbf{Y}[:,w] + \textbf{U}_{c}\textbf{h}_{w-1})
\] 
The operators of forget gate $\textbf{W}_{g}$ and $\textbf{U}_{g}$ control the deletion of information from the\textit{ previous} sequence. The output of $g_{w}$ is between 0 (delete the information) and 1 (keep the information). 
\[
\textbf{g}_{w} = \sigma(\textbf{W}_{g}\textbf{Y}[:,w] + \textbf{U}_{g}\textbf{h}_{w-1})
\] 
The operators of update gate $\textbf{W}_{i}$ and $\textbf{U}_{i}$ control the addition of information from the \textit{current} sequence. The output of $i_{w}$ is also between 0 and 1.
\[
\textbf{i}_{w} = \sigma(\textbf{W}_{i}\textbf{Y}[:,w] + \textbf{U}_{i}\textbf{h}_{w-1})
\] 
The operators $\textbf{W}_{o}, \textbf{U}_{o}, \textbf{W}_{i}, \textbf{U}_{i}, \textbf{W}_{g}, \textbf{U}_{g}, \textbf{W}_{c}$ and $ \textbf{U}_{c}$ are solved by training the RNN.
\bigskip

\noindent To get a fixed size temporal approximation, the RNN should project all the input sequence to a single output. This architecture of RNN is called \textit{Sequence to One} RNN. 
\begin{algorithm}
  \caption{$\textbf{f}$ = $Seq2One\_LSTM$($\textbf{Y}$) }\label{alg:s2olstm}
  \begin{algorithmic}[1]
    \Statex \textbf{Input :} $\textbf{Y} \in \mathbb{R}^{T \times W}$
    \Statex \textbf{Output :} $\textbf{f} \in \mathbb{R}^{Z}$
    \State Initialize $\textbf{h}_{0}$
    \For{$ w \in \{1,2,..,W\}$}
    \State  $\textbf{h}_{w} \leftarrow \bm{\Theta}_{LSTM}(\textbf{Y}[:,w],\textbf{h}_{w-1})$
    \EndFor
    \State $\textbf{f} = \textbf{W}\textbf{h}_{W}$
  \end{algorithmic}
\end{algorithm}
\FloatBarrier

\noindent Often times, multiple layers of RNN are used. An illustration of  two layer LSTM with a \textit{sequence to sequence} LSTM in between is shown in algorithm \ref{alg:2lstm}. The input sequence transformed in to a hidden sequence, which is then projected to a single output by the second layer. The functions in between ($\Phi_{1}, \Phi_{2}$) represents the transition operation between the layers. Several non-linear activations and transition operations have been developed to address the problems while training deep neural network (see Sec. \ref{training}). 
  
\begin{minipage}[t]{7.5cm}
  \vspace{0pt}  
\begin{algorithm}[H]
  \caption{$\textbf{f}$ = $LSTM2(\textbf{Y})$}\label{alg:2lstm}
  \begin{algorithmic}[1]
    \Statex \textbf{Input :} $\textbf{Y} \in \mathbb{R}^{T \times W}$
    \Statex \textbf{Output :} $\textbf{f} \in \mathbb{R}^{Z}$
    \Statex
    \Statex
    \Statex
    \State $\textbf{F}_{1} = \Phi_{1}(Seq2Seq\_LSTM(\textbf{Y}))$
    \State $\textbf{f} = \Phi_{2}(Seq2One\_LSTM(\textbf{F}_{1})$
  \end{algorithmic}
\end{algorithm}
\end{minipage}%
\begin{minipage}[t]{7.5cm}
  \vspace{0pt}
\begin{algorithm}[H]
  \caption{$\textbf{F}$ = $Seq2Seq\_LSTM$($\textbf{Y}$) }\label{alg:s2slstm}
  \begin{algorithmic}[1]
    \Statex \textbf{Input :} $\textbf{Y} \in \mathbb{R}^{T \times W}$
    \Statex \textbf{Output :} $\textbf{F} \in \mathbb{R}^{Z \times W}$
    \State Initialize $\textbf{h}_{0}$
    \For{$ w \in \{1,2,..,W\}$}
    \State  $\textbf{h}_{w} \leftarrow \bm{\Theta}_{LSTM}(\textbf{Y}[:,w],\textbf{h}_{w-1})$
    \State $\textbf{F}[:,w] = \textbf{W}\textbf{h}_{W}$
    \EndFor
  \end{algorithmic}
\end{algorithm}
\end{minipage}
\FloatBarrier

\section{Multi-label Classifier}
\label{classifier}
The classifier takes the feature $\textbf{f}$ as input and performs the classification task. A \textit{discriminative} \footnote{The classification model can either be \textit{generative} (approximates class distribution) or \textit{discriminative} (approximates class boundaries). For a binary classifier, the classes are either 0 or 1} binary classifier can be formalised as, 
\[
\ell_{i} = b(\zeta_{i}) =
\begin{cases}
1, & \text{if  } \zeta_{i} > \epsilon \\
0, & \text{otherwise}
\end{cases}
\qquad i \in \{1,2,..,L\}, \ell_{i} \in \{0,1\}, \textbf{f} \in \mathbb{R}^{Z}
\]
where $b$ is a binary classifier, given the feature vector $\textbf{f}$. The output of a binary classifier $\ell_{i}$ is either 0 or 1. There are $L$ binary classifier outputs for $L$ labels. $\zeta_{i}$ is the $i^{th}$ output of the classification function $C$ and the classifier output $\ell_{i}$ is 1 if $\zeta_{i}$ is greater than certain threshold $\epsilon$ 
\[
\bm{\zeta} = C(\textbf{f}) \qquad \bm{\zeta} \in \mathbb{R}^{L}
\]
The final prediction ($\textbf{pred}$) is an index set of all classifier outputs that is equal to 1 
\[
\textbf{pred} = \{\ell_{i} | \ell_{i} = 1 \}
\]
The following equivalent notation will be used in algorithms of upcoming chapters 
\[
\textbf{pred}_{(\epsilon)} = \{ b(\zeta_{i}) | b(\zeta_{i}) = 1 \} \qquad i \in \{1,2,..,L\}
\]
The classification function $C$ projects the feature $\textbf{f}$ to the vector $\bm{\zeta}$ in the label space. Depending on how this function is defined, there are several discriminative classifiers like \textit{multi-layer perceptrons}, \textit{support vector machines}, \textit{random-forest}. In this thesis, only \textit{multi-layer perceptron} is considered.


\subsection{Two-layer perceptron}
A two layer perceptron first projects the features $\textbf{f}$ to a hidden layer $\textbf{h}$ and some element-wise non-linear operation $\Phi$ is applied.
\[
\textbf{h} = \Phi(\textbf{W}_{L1}\textbf{f}) \qquad \textbf{h} \in \mathbb{R}^{L1}, \textbf{W}_{L1} \in \mathbb{R}^{L1 \times Z}
\]
Without the non-linear operation $\Phi$, the operators $\textbf{W}_{L1}$ and $\textbf{W}$ can multiply out to generalize a single layer perceptron. The motivation for using a two layer perceptron is because single layer perceptron can approximate only linear class boundries. But \textit{universal approximation theorem}\cite{upt} states that a single hidden layer with some non-linear activation function can approximate any function. The choice of $\Phi$ will be explained in section \ref{training}. The second operator $\textbf{W}$ performs the projection from hidden vector $\textbf{h}$  
\[
\bm{\zeta} = \sigma (\textbf{W}\textbf{h}) \qquad \bm{\zeta} \in \mathbb{R}^{L}, \textbf{W} \in \mathbb{R}^{L \times L1}
\]
where $\sigma$ is a sigmoid function, which projects the output of second layer between 0 and 1. 
\begin{equation}
\label{sigmoid}
\sigma (x) = \frac{1}{1 + e^{-x}}
\end{equation}
Thus $0 \leq \zeta_{i} \leq 1$ and threshold $\epsilon$ can be set in-between 0 and 1 and eventually the final prediction $\textbf{pred}$ can be computed. 

\section{Training}
\label{training}
The iterative steps involving the computation of the operators that is optimal for transformations for our context-based classification task is called \textit{training}. When we train only the operators of classifying function $C$ (that is, $\textbf{W}$ and $\textbf{W}_{L1}$), the classification performance will only be as good as the information encoded in the feature $\textbf{f}$. Let us assume that the features $\textbf{f}$ obtained as a result of transformations $R$, $D$, and $T$ is optimal for the task at hand. Hence, only the classifier $C$ is trained. The abstract prediction model is shown in algorithm \ref{abstraction}.
\begin{algorithm}
  \caption{$\textbf{pred}$ = $Model$($\textbf{a}$) }\label{abstraction}
  \begin{algorithmic}[1]
    \Statex \textbf{Input :} $\textbf{a} \in \mathbb{R}^{N}$
    \Statex \textbf{Output :} $\textbf{pred}$ \Comment{indices of predicted labels}
    \State $\textbf{X} = R(\textbf{a})$ \Comment{$\textbf{X} \in \mathbb{R}^{R \times P}$}
    \State $\textbf{Y} = D(\textbf{X})$ \Comment{$\textbf{Y} \in \mathbb{R}^{T \times W}$}
    \State $\textbf{f} = T(\textbf{Y})$ \Comment{$\textbf{f} \in \mathbb{R}^{Z}$}
    \State $\bm{\zeta} = C(\textbf{f} \quad |\textbf{W},\textbf{W}_{L1})$ \Comment{$\bm{\zeta} \in \mathbb{R}^{L}$}
    \State $\textbf{pred} = \{ b(\zeta_{i}) | b(\zeta_{i}) = 1 \}$ \Comment{$ i \in \{1,2,..,L\}, b( \zeta_{i}) \in \{0,1\}$}
  \end{algorithmic}
\end{algorithm}
\FloatBarrier
\noindent Considering a two layer perceptron for classification, $\bm{\zeta}$ is computed as,
\begin{equation}
\label{eq:mlp}
\bm{\zeta} = \sigma ( \textbf{W} \Phi (\textbf{W}_{L1}\textbf{f}))
\end{equation}
\noindent Computing $\textbf{W}$ and $\textbf{W}_{L1}$ is an inverse problem. That is, we need a target $\textbf{t}$ that approximates $\bm{\zeta}$ for true classifications from a set of observations. To do this, recall that $\bm{\zeta}$ is a $L$ dimensional vector and $L$ is the number of labels in consideration. $\zeta_{i}$ is the classifier output for $i^{th}$ label and $i \in \{1,2,..,L\}$. With sigmoid ($\sigma$) projection we know that $0 \leq \zeta_{i} \leq 1$. If we assume $t_{i}$ equal to 1 for true classifications and 0 for the rest, then the transformation operators ($\textbf{W}$ and $\textbf{W}_{L1}$) can be solved such that the classifier output $\zeta_{i}$ is close to 1 for true classifications.

\subsection{First-order gradient descent}
Solving for the operators by \textit{first-order gradient descent} involves the following steps : 
\begin{enumerate}
\setlength\itemsep{0em}
\item Initialize $\textbf{W}, \textbf{W}_{L1}$
\item Run the model for a sample and compute loss $E$ = $loss(\bm{\zeta}, \textbf{t})$
\item Compute the gradient of loss with respect to the operators ($ \frac{\partial E}{\partial \textbf{W}}, \frac{\partial E}{\partial \textbf{W}_{L1}}$)
\item Update $\textbf{W}$ and $\textbf{W}_{L1}$
\item Recompute loss, gradients and update the parameters until convergence.
\end{enumerate}

\subsubsection{Loss function}
To train a classifier, a loss function (or error function) have to be defined. The \textit{least square} error function is sensitive to outliers in the training data. Hence \textit{cross-entropy}\cite{ml} loss function is considered. Error function defined as the negative log-likelihood is the \textit{cross-entropy} error function. \textit{Likelihood} that the parameters ($\textbf{W}, \textbf{W}_{L1}$) approximate a set of targets for label $i$ for $N$ training samples $({t_{i}}^{1}, {t_{i}}^{2},...,{t_{i}}^{N})$ is
\[
\mathcal{L} (\textbf{W}, \textbf{W}_{L1} | {t_{i}}^{1}, {t_{i}}^{2},...,{t_{i}}^{N} ) =  \displaystyle\prod_{n=1}^{N} {\zeta_{i(n)}}^{{t_{i}}^{(n)}}{(1-\zeta_{i(n)})}^{1 - {t_{i}}^{(n)}} \qquad t_{i} \in \{0,1\}
\]
where the \textit{likelihood} is 1, as the classifier output $\zeta_{i}$ approaches the target $t_{i}$. The log-likelihood is taken to get rid of the multiplication that would cause numerical problems over large $N$. The negative of the log-likelihood is taken to pose the optimization as a \textit{minimization} problem. Therefore, minimizing the \textit{cross-entropy} loss for label $i$ is equivalent to minimizing the \textit{negative log likelihood}
\[
\text{Minimize } E_{i}  = -ln(\mathcal{L}) = - \displaystyle\sum_{n=1}^{N} \{ {t_{i}}^{(n)} ln \zeta_{i(n)} + (1-{t_{i}}^{(n)}) ln (1-\zeta_{i(n)}) \}
\]
For $L$ labels, the total loss is minimized,
\[
\text{Minimize } \displaystyle\sum_{i=1}^{L}E_{i} 
\]
Moreover, with gradient descent optimization, the loss is minimized for a batch of samples ($B$) for every iteration. $B$ is called the \textit{batch-size}. Thus, loss for every iteration would be,
\[
{E_{i}}^{(B)} = - \displaystyle\sum_{i=1}^{L}\displaystyle\sum_{n=1}^{B} \{ {t_{i}}^{(n)} ln \zeta_{i(n)} + (1-{t_{i}}^{(n)}) ln (1-\zeta_{i(n)}) \}
\]

\subsubsection{Computing gradients}
After every iteration, the gradient of the total loss ($E$) with respect to the parameters ($\textbf{W}, \textbf{W}_{L1}$) have to computed. This gradient will then be used to update the parameters for next iteration. The gradients are computed by applying the chain rule. An illustration for computing the gradients for the two layer perceptron is shown
\begin{equation}
\label{grad1}
\frac{\partial E}{\partial \textbf{W}} = \frac{\partial E}{\partial \bm{\zeta}}  \frac{\partial \bm{\zeta}}{\partial \sigma}  \frac{\partial \sigma}{\partial \textbf{W}}
\end{equation}
\begin{equation}
\label{grad2}
\frac{\partial E}{\partial \textbf{W}_{L1}} = \frac{\partial E}{\partial \bm{\zeta}}  \frac{\partial \bm{\zeta}}{\partial \sigma}  \frac{\partial \sigma}{\partial \Phi} \frac{\partial \Phi}{\partial \textbf{W}_{L1}}
\end{equation}
Efficient computation of the gradients is achieved with \textit{back-propagation} algorithm. To ease the computations, the non-linearities and loss functions are usually chosen in such a way that the variables required for gradient computation are known in the forward pass.

\subsubsection{Updating the parameters}
Standard stochastic gradient descent update for every iteration is,
\[
\textbf{W} = \textbf{W} - \eta \frac{\partial E}{\partial \textbf{W}}
\]
where $\eta$ is the learning rate. But choosing a proper learning rate is difficult because if $\eta$ is too small, convergence can be too slow, while $\eta$ that is too high can hinder convergence. Additionally, the same learning rate is applied to all parameters across the layers. Moreover, the neural networks lead to highly non convex optimization problem and to avoid getting trapped in a local optima is challenging. Therefore, several optimization algorithms specialized for neural network training emerged to deal with the aforementioned challenges. In this thesis, Adaptive Moment (ADAM) Estimation updates\cite{adam} will be used. ADAM uses the exponentially decaying average of past moments (first moment $m$ and second moment $v$) and squared gradients ($g$) to update the parameters. Update for iteration $t$ is,
\[
m_{t} = \beta_{1}m_{t-1} + (1- \beta_{1})g_{t}
\]
\[
v_{t} = \beta_{2}v_{t-1} + (1-\beta_{2}){g_{t}}^{2}
\]
As $m_{t}$ and $v_{t}$ are initialized as vectors of 0's, the authors of ADAM observe that they are biased towards zero, especially during the initial time steps, and especially when the decay rates are small (i.e. $\beta_{1}$ and $\beta_{2}$ are close to 1). They counteract these biases by computing bias-corrected first and second moment estimates\cite{adam_o}:
\[
{m}_{t} = \frac{m_{t}}{1-{\beta_{1}}^{t}}
\]
\[
{v}_{t} = \frac{v_{t}}{1-{\beta_{2}}^{t}}
\]
Thus, update of each parameter for the next iteration is,
\[
W_{t+1} = W_{t} - \frac{\eta}{\sqrt{v_{t}} + \epsilon}m_{t}
\]
The authors propose default hyper-parameter values of 0.9 for $\beta_{1}$, 0.999 for $\beta_{2}$, and ${10}^{-8}$ for $\epsilon$

\subsection{Deep learning}
As mentioned before, if we train only the operators of classifying function $C$, then the classification performance will only be as good as the information encoded in the feature $\textbf{f}$. But, if it is possible to solve for the transformation operators that compute the features $\textbf{f}$, then it is possible to obtain features optimal for the considered task. That is, if we push the supervision into the temporal approximator function $T$ with a \textit{sequence to one} LSTM, then the classifier performance is no longer limited by the encodings in the feature $\textbf{f}$. Because, now the operators of RNN can be solved for optimality in addition to the operators of 2 layer perceptron, and therefore $\textbf{f}$ can be optimal for the task. Now we have to update every iteration not just $\textbf{W}, \textbf{W}_{L1}$, but also $\textbf{W}_{o}, \textbf{U}_{o}, \textbf{W}_{i}, \textbf{U}_{i}, \textbf{W}_{g}, \textbf{U}_{g}, \textbf{W}_{c}$ and $ \textbf{U}_{c}$   

\begin{algorithm}
  \caption{$\textbf{pred}$ = $Model$($\textbf{a}$) }\label{abstraction2}
  \begin{algorithmic}[1]
    \Statex \textbf{Input :} $\textbf{a} \in \mathbb{R}^{N}$
    \Statex \textbf{Output :} $\textbf{pred}$ \Comment{indices of predicted labels}
    \State $\textbf{X} = R(\textbf{a})$ \Comment{$\textbf{X} \in \mathbb{R}^{R \times P}$}
    \State $\textbf{Y} = D(\textbf{X})$ \Comment{$\textbf{Y} \in \mathbb{R}^{T \times W}$}
    \State $\textbf{f} = Seq2One\_LSTM(\textbf{Y} \quad | \textbf{W}_{k}, \textbf{U}_{k})$ \Comment{$ k = \{ i_{r},o,g,c\}, \textbf{f} \in \mathbb{R}^{Z}$}
    \State $\bm{\zeta} = C(\textbf{f} \quad |\textbf{W},\textbf{W}_{L1})$ \Comment{$\bm{\zeta} \in \mathbb{R}^{L}$}
    \State $\textbf{pred} = \{ b(\zeta_{i}) | b(\zeta_{i}) = 1 \}$ \Comment{$ i \in \{1,2,..,L\}, b( \zeta_{i}) \in \{0,1\}$}
  \end{algorithmic}
\end{algorithm}
\FloatBarrier
\noindent However, the performance is still limited by the frame-wise features $\textbf{Y}$. But the supervision can be further pushed inside by using convolution neural networks and solving for it's operators ($\textbf{W}_{C1}, \textbf{W}_{C2}, \textbf{W}_{C3}$) in addition to the operators of RNN and perceptron.
\begin{algorithm}
  \caption{$\textbf{pred}$ = $Model$($\textbf{a}$) }\label{abstraction3}
  \begin{algorithmic}[1]
    \Statex \textbf{Input :} $\textbf{a} \in \mathbb{R}^{N}$
    \Statex \textbf{Output :} $\textbf{pred}$ \Comment{indices of predicted labels}
    \State $\textbf{X} = R(\textbf{a})$ \Comment{$\textbf{X} \in \mathbb{R}^{R \times P}$}
    \State $\textbf{Y} = CNN(\textbf{X} \quad | \textbf{W}_{C1}, \textbf{W}_{C2}, \textbf{W}_{C3})$ \Comment{$\textbf{Y} \in \mathbb{R}^{T \times W}$}
    \State $\textbf{f} = Seq2One\_LSTM(\textbf{Y} \quad | \textbf{W}_{k}, \textbf{U}_{k})$ \Comment{$ k = \{ i_{r},o,g,c\}, \textbf{f} \in \mathbb{R}^{Z}$}
    \State $\bm{\zeta} = C(\textbf{f} \quad |\textbf{W},\textbf{W}_{L1})$ \Comment{$\bm{\zeta} \in \mathbb{R}^{L}$}
    \State $\textbf{pred} = \{ b(\zeta_{i}) | b(\zeta_{i}) = 1 \}$ \Comment{$ i \in \{1,2,..,L\}, b( \zeta_{i}) \in \{0,1\}$}
  \end{algorithmic}
\end{algorithm}
\FloatBarrier
\noindent From the illustration in algorithm \ref{MFCC_learn}, it can be seen that the learning problem can pushed up to the point of even replacing the $STFT$ operators. That is, the model can be now trained to detect the pattern directly from the raw audio signal for our classification task. Since the training data is now able to affect the operators of feature computations, the context of learning problem is now called \textit{deep learning} 

\subsubsection{Issues}
\label{issues}
As exciting as it may sound, deep learning has two major issues,\\
\\
\textit{\textbf{Requires large training data} :}\\
\\
Looking at all the application domains where deep learning is successful (image /speech recognition), they are the ones where acquiring a lot of data is feasible. As the number of parameters to optimize increases, not only that more iterations are needed to converge, but there is also a risk that the solutions converge to learning the noise in data. This is called \textit{over-fitting}.\\
\\
\textit{Transfer-learning :} This is one way to address this issue for smaller datasets. \textit{Transfer learning} is possible only when an alternate large dataset for similar task (\textit{source task}) is available. It is  then possible to train with the smaller dataset by initializing the weights with the values converged in the source task. This is called \textit{fine-tuning} the model.\\
\\
\textit{Drop-out :} This is a regularizer that counter over fitting by randomly setting parameters to zero. This is called \textit{dropping} the connection. At each training iteration, the connections can be \textit{dropped out} with probability $p$. ($Drop_{(p)}$)\\  
\\
\textit{\textbf{Vanishing gradients} :}\\  
\\
As the number of layers in the neural network increases, the risk of gradients approaching zero in the earlier layers increases. This will increase the number of iterations required for convergence. As an illustration, looking at the gradient computation equations for the final layer in equation \ref{grad1} and the penultimate layer in equation \ref{grad2}, it can be seen that as we move deeper, the number of multiplications in the chain rule required for calculating the gradient increases. If one of those gradients in chain have a value close to zero, then the gradient with respect to the parameters will also be close to zero. Thus, the non-linearities $\Phi$ are chosen such that the gradient is boosted. The non-linearities - Rectified linear units ($ReLU$)\cite{relu} and Exponential linear units ($ELU$)\cite{elu} have been used in this thesis.  
\[
ReLU(x) = max(0,x)
\]     
\[
ELU(x) = 
\begin{cases}
x, & \text{if } x \geq 0 \\
a(e^{x}-1), & \text{otherwise}
\end{cases}
\]
where $a > 0$ is a hyper-parameter. 
\chapter{Review of literature and Model Selection } % Main chapter title

\label{Chapter3} % For referencing the chapter elsewhere, use \ref{Chapter2} 

Using content-based music information for solving several music information retrieval tasks is not new, but a decade long research efforts have been put. Hunting for the right model for our task and to justify it to be superior to the rest requires thorough understanding of evolution of such techniques. In section 3.1, the dynamics of the literature that has lead to the use of deep learning techniques for MIR tasks have been discussed. In section 3.2, the inferences from state of art techniques have been used to short list models for the experiments.       


\section{Evolution of algorithms}
A number of surveys (e.g. [8, 22, 29]) amply document what is a decades-long research effort at the intersection of music, machine learning and signal processing. In a broader sense, all techniques have a two-stage architecture: first, features are extracted from music audio signals to transform them into a more meaningful representation. These features are then used as input to a classifier, which is trained to perform the task at hand. This dedicated analysis for music features emerged due to the fact that music signals possess specific acoustic and structural characteristics that distinguish them from spoken language or other non musical signals. The motivation for writing this section elaborately is to make the answers for following questions clear,
\begin{enumerate}[label=(\alph*)]
\setlength\itemsep{0em}
\item To approach the solution, should we look for better classifier or better feature extractor?  
\item Can neural networks outperform features extracted through \gls{basis transformation} approaches on a medium sized dataset. What is the trade off between the both?
\item If deep learning had to be used, does fine-tuning (transfer learning) work for multi-label classification task? Are there any pre-trained models already available?
\end{enumerate}

\subsection{From classifier to feature emphasis}
Looking back to our history before 2010, there is a clear trend in MIR of applying increasingly more powerful machine learning algorithms to the same feature representations to solve a given task. There are also ample surveys with evidence suggesting that appropriate feature representations significantly reduce the need for complex semantic interpretation methods [2]. Evidence from genre classification and chord recognition task have been discussed below. 

\subsubsection{Audio music genre classification using different classifiers and feature selection methods. Proceedings of the International Conference on Pattern Recognition, Hong-Kong, China,2006}
Fixing the features, ten different classifiers were compared, namely: Fisher (Fisher classifier), LDC (Linear classifier assuming normal densities with equal covariance matrices), QDC (Quadratic classifier assuming normal densities), UDC (Quadratic classifier assuming normal uncorrelated densities), NBC (Naïve Bayes Classifier), PDC (Parzen Density Based Classifier), KNN (Knearest neighbor with optimal k computed using leave one out cross validation), KNN1 (1 nearest neighbor), KNN3 (3 nearest neighbor), KNN5.  It is seen that a ceiling performance of 80\% accuracy on GTZAN dataset was obtained by using combination of classifiers and squeezing every last percentage from the same features. This suggested the need for robust feature representation for further improvements.

\subsubsection{Exploring common variations in state of the art chord recognition systems. In Proc. SMC, 2010.}
The significance of robust feature representations was demonstrated by using appropriate filtering of chroma features to increase system performance even for the simplest classifier. An overall reduction of performance variation across all classifiers was also shown[9].

\subsection{From hand-crafting to feature learning}
 
Feature learning consists of exploiting the structure of the \textit{data distribution} to construct a new representation of the input.Although MFCC are hand-crafted, they pose a tough competition, which makes researchers not to ignore them completely. Recalling that feature extractors can be used in hierarchy (see section ??), there is a chance that MFCCs can outperform for some combination of feature learning at higher levels. It is also worthy to consider MFCCs because of computational efficiency. Aggregating hand-crafted features for music tagging was introduced in [25]. Several subsequent works rely on a Bag of frames approach - where a collection of features are computed for each frame and then statistically aggregated. Typical features are designed to represent physical or perceived aspects of sound and include MFCCs, MFCC derivatives and spectral centroids.
\bigskip

\noindent Although MFCC related features work great for speech recognition tasks, it falls short in performance for MIR tasks[ ]. This is especially because long range temporal structure is crucial in music (MFCC derivatives only encode short term temporal information). In [ ], better performance was achieved by using different scales of PCA whitened frames, and achieves state of art result for multi label classification on MTT dataset till date.


\subsubsection{[2011] Multi label class : temporal pooling auc 86  [ mfcc ]}

The pipe line of their algorithm is shown below . The formalism of the notations used are consistent with explanations in chapter 2. The PCA whitened mel-power spectrogram is compared with MFCC features on Magna tag a tune dataset. It was shown that the former achieve a performance of AUC 0.87 out performing MFCCs which was 0.77.  
\bigskip

\noindent Signal ($\textbf{a}$) is sampled at 22.1 KHz. Then STFT with window length 1024 and stride 512 is computed with FFT algorithm. This is followed by conversion to mel power-spectrogram with 128 bins, followed by PCA Whitening which selects the top 120 variant frequencies. Another transformation is done by stacking a single layer perceptron ($L(\textbf{W})$ means the weight matrix $\textbf{W}$ is learned by training a neural network (see section ??). The temporal pooling is done by summarizing every 2.3s frame with suitable functions (see [ ] for details). The matrix $\textbf{W}_{1}$ learns the optimal features for pooling. The resulting feature is then classified by two layer perceptron with 1000 hidden units with sigmoid ($\sigma$) activations.

\begin{algorithm}
  \caption{$Pred$ = MODEL($\textbf{a}$) }\label{Temporal Pooling}
  \begin{algorithmic}[1]
    \Statex \textbf{Input :} $\textbf{a} \in \mathbb{R}^{N}$
    \Statex \textbf{Output :} $Pred \in \mathbb{R}^{L}$ 
    \State $\textbf{C} = STFT(\textbf{a})$ \Comment{$\textbf{C} \in \mathbb{C}^{M \times P}$}
    \State $\textbf{Y}_{r} = \textbf{C} \odot \textbf{C}$ \Comment{$\textbf{Y}_{r} \in \mathbb{R}^{M \times P}$}
    \State $\textbf{R} = MEL(\textbf{Y}_{r})$ \Comment{$\textbf{R} \in \mathbb{R}^{128 \times P}$}
    \State $\textbf{X}_{1} = PCA\_WHITEN(\textbf{R})$ \Comment{$\textbf{X}_{1} \in \mathbb{R}^{120 \times P}$}
    \State $\textbf{X}_{2} = L(\textbf{W}_{1})\textbf{X}_{1}$  \Comment{$\textbf{W}_{1} \in \mathbb{R}^{S \times 120}, \textbf{X}_{2} \in \mathbb{R}^{S \times P}$}
    \State $\textbf{y} = POOL(\textbf{X}_{2})$ \Comment{$\textbf{y} \in \mathbb{R}^{S.W}$}
    \State $Pred = \sigma(L(\textbf{W}_{3})\sigma(L(\textbf{W}_{2})\textbf{y}))$ \Comment{$\textbf{W}_{2} \in \mathbb{R}^{1000 \times S.W}, \textbf{W}_{3} \in \mathbb{R}^{L \times 1000}$}
  \end{algorithmic}
\end{algorithm}
\noindent It is important to note that this algorithm is does not work on audio of arbitrary length because of their design of temporal pooling (because fixed sized features are needed for classification).

\subsubsection{[2012] Multi scale : spec : auc 89.8 (still not deep learning, feature learning is not task specific)}
The result reported by this model is the current state-of-art on MTT dataset (AUC 0.898). Here, reducing the mel spectrogram to different sizes is done in parallel by \gls{gaussian pyramids}. The resulting features are then concatenated. This was mainly done to see the relevance of rhythmic structure from longer time-scales. PCA whitened frames in the mel-spectrogram are subjected to unsupervised learning with K-Means. It has been shown that learning features at larger timescales in addition to short time scales improves performance. This also suggests the existence of periodicity at longer timescales emerging from rhythmic structure, repeated motifs and musical form.  


\begin{algorithm}
  \caption{$Pred$ = MODEL($\textbf{a}$) }\label{Temporal Pooling}
  \begin{algorithmic}[1]
    \Statex \textbf{Input :} $\textbf{a} \in \mathbb{R}^{N}$
    \Statex \textbf{Output :} $Pred \in \mathbb{R}^{L}$ 
    \State $\textbf{C} = STFT(\textbf{a})$ \Comment{$\textbf{C} \in \mathbb{C}^{M \times P}$}
    \State $\textbf{Y}_{r} = \textbf{C} \odot \textbf{C}$ \Comment{$\textbf{Y}_{r} \in \mathbb{R}^{M \times P}$}
    \State $\textbf{R} = MEL(\textbf{Y}_{r})$ \Comment{$\textbf{R} \in \mathbb{R}^{R \times P}$}
    \For{$i \in \{1,..,W\}$}
     \State $\textbf{X}_{1} \leftarrow GAUSSIAN\_PYRAMID(\textbf{R},i)$ \Comment{$\textbf{X}_{1} \in \mathbb{R}^{R \times Q1_{i}}$}
    \State $\textbf{X}_{2} \leftarrow PCA\_WHITEN(\textbf{X}_{1})$ \Comment{$\textbf{X}_{2} \in \mathbb{R}^{S1 \times Q1_{i}}$}
     \State $\textbf{X}_{3} \leftarrow K\_MEANS(\textbf{X}_{2},J)$ \Comment{$\textbf{X}_{3} \in \mathbb{R}^{S2 \times Q2_{i}}$}
    \State $\textbf{Y}[i] \leftarrow MAX\_POOL(\textbf{X}_{3})$ \Comment{$\textbf{Y}[i] \in \mathbb{R}^{S2}, \textbf{Y} \in \mathbb{R}^{S2 \times W}$}
    \EndFor
    \State $\textbf{y} = FLATTEN(\textbf{Y})$ \Comment{$\textbf{y} \in \mathbb{R}^{S2.W}$}
    \State $Pred = \sigma(L(\textbf{W}_{3})ReLU(L(\textbf{W}_{2})\textbf{y}))$ \Comment{$\textbf{W}_{2} \in \mathbb{R}^{1000 \times S2.W}, \textbf{W}_{3} \in \mathbb{R}^{L \times 1000}$}
  \end{algorithmic}
\end{algorithm}
\noindent The take-away is that modelling relation between features at rhythmic intervals does help.

\subsection{Transfer Learning by supervised pre-training}
Deep learning and feature learning techniques  typically require large amounts of training data to work well. The following publication propose to exploit models trained on larger datasets like Million Song Dataset and use those weights as initialization for classification on smaller datasets like GTZAN. This is called supervised pre-training and it is essential to have a source task that requires a very rich feature representation, so as to ensure that the information content of this representation is likely to be useful for other tasks

\subsubsection{[2014] Transfer learning : auc 88.0}
This model achieves AUC 0.88 on MTT dataset. The workflow for source and target are shown below,
\begin{figure}[h] 
\centering
\includegraphics[width=0.95\textwidth]{specLeak1}
\caption{Schematic overview of the workflow of transfer learning}
 \label{fig:transfer learning}
 \end{figure}
\FloatBarrier
\bigskip

\noindent \textbf{Source task}: 
The low-level features from audio spectrograms are learned through unsupervised learning by spherical K-Means. To tackle problems created by redundant and sparse labels, dimensionality reduction is done in the label space using PCA. The model is then trained to predict the reduced label representation.
\bigskip

\noindent \textbf{Target task}
Next,  the trained models are used to extract higher-level features from other datasets, which are then passed to train shallow classifiers for different but related target tasks. This workflow is visualized in fig[ ] Dashed arrows indicate transfer of the learned feature extractors from the source task to the target task.
\bigskip

\noindent It has been shown that features learned in this fashion work well for auxiliary audio classification tasks on different datasets, consistently outperforming a purely unsupervised feature learning approach.

\subsection{Towards deep learning}
As alternative to the above systems, Deep Neural Networks(DNN) have recently become widely used in audio analysis, following their success in computer vision, speech recognition [19]. From an engineering perspective, DNNs sidestep the problem of creating or finding audio features relevant to a task. Their general structure includes multiple hidden layers with hidden units trained to represent some underlying structure in data.

\chapter{Experiments and Results} % Main chapter title

\label{Chapter4} % For referencing the chapter elsewhere, use \ref{Chapter2} 

The aim of this thesis is to find the optimal algorithm for content-based multi-label classification of music tracks, that can be solved with minimal training data. In Chapter 3, the state of art models were reviewed and in section \ref{model}, the short-comings of these algorithms in addressing the problems discussed in Chapter 1 (see \ref{problems}) were pointed out. In this chapter, the experiments that will lead to finding the best algorithm using the components short-listed in \ref{model} will be described.

\section{Dataset and Evaluation}
\label{dataset}

\section{Experiments}
\label{experiments}

\section{Summary of Results}
\label{results}
\externaldocument{chapter1}
\externaldocument{chapter2}
\chapter{Conclusion} % Main chapter title

\label{Chapter5}
The performance of music tagging algorithm will be higher if the extracted features encode the acoustic cues necessary for discriminating the semantics. Supervised feature learning is used to obtain features that could be optimal for our context of music tagging. Convolution and recurrent neural networks were used to induct supervised learning into the feature extraction pipeline. The number of parameters to solve increases as the learning problem becomes deeper and hence more training data is required for convergence. Therefore, in this thesis, transfer learning with models trained on \textit{MagnaTagaTune} dataset were analysed by comparing different levels of fine-tuning on the target dataset. It has been found that fine-tuning entire CNN + RNN on target data boosts the performance and there were significant performance difference with black-box CNN. It would be useful to bridge this gap to improve performance, because deep levels of training on our target data, which is small, struggles to out-perform MFCC features.     

\subsubsection{Proposal for improvements}
\begin{itemize}
\setlength\itemsep{0em}
\item Since MFCC performs better than fine-tuned CNNs, it would be worthy to analyse CNN architectures trained on the source data by initializing CNN filters with cosine functions to get richer features. (In chapter 2, section \ref{stacked}, an illustration of MFCC computation as convolution operations was shown)
\item Singular value decomposition of the converged filters at each CNN layer can be performed to see if they are converging to any known mathematical function. If it does, then it would be useful to set those functions as initialization before training.   
\end{itemize}
 


% Appendix A

%\chapter*{Frequently Asked Questions} % Main appendix title

%\label{AppendixA} % For referencing this appendix elsewhere, use \ref{AppendixA}
\begin{appendices}

\chapter{}
\section{Basis Transformation}
Any vector $\textbf{y} = [y_{1}, y_{2},...,y_{n}] \in V$ can be represented as a \gls{linear combination} of it's \gls{basis} $\textbf{V} = [\textbf{v}_{1}, \textbf{v}_{2},...,\textbf{v}_{n}] \in \mathbb{C}^{N \times N}$
  
\[
\textbf{y} = \displaystyle\sum_{i=1}^{N}\alpha_{i}\textbf{v}_{i} = \textbf{V}^{T}\bm{\alpha}
\]
Where $\alpha_{i}$ are the coefficients or coordinates with respect to $\textbf{v}_{i}$. If there exists another set of \gls{basis} vectors $\textbf{U} = [\textbf{u}_{1}, \textbf{u}_{2},...,\textbf{u}_{m}] \in \mathbb{C}^{M \times M}$, then we can represent $\textbf{y}$ as a \gls{linear combination} of $M$ \gls{basis}.
\[
\textbf{y} = \displaystyle\sum_{i=1}^{M}\beta_{i}\textbf{u}_{i} = \textbf{U}^{T}\bm{\beta}
\]
Where $\beta_{i}$ are the coefficients or coordinates with respect to $\textbf{u}_{i}$.
\bigskip

\noindent For a vector in basis $\textbf{V}$ with co-ordinates $\bm{\alpha}$, \textbf{basis transformation} is defined as representing the same vector with the new co-ordinates $\bm{\beta}$ in basis $\textbf{U}$.
\bigskip

\noindent Cartesian basis ($\textbf{e}_{i}$) are standard basis vectors with $\bm{\alpha} = \textbf{y}$
\[
\textbf{y} = \displaystyle\sum_{i=1}^{N}y_{i}\textbf{e}_{i} = \displaystyle\sum_{i=1}^{M}\beta_{i}\textbf{u}_{i}
\]

\end{appendices}
% Appendix A

%\chapter*{Frequently Asked Questions} % Main appendix title

%\label{AppendixA} % For referencing this appendix elsewhere, use \ref{AppendixA}
\begin{appendices}

\chapter{Experiments}
\section{Validation Tags}
The list of tags and it's number of occurrences in the validation set.\\
\\
\begin{tabular}{| p{.05\textwidth} | p{.20\textwidth} | p{.05\textwidth}|}
\hline
\textbf{No.} & \textbf{Tag} & \textbf{Freq}\\
\hline
1 & down beat & 3\\
\hline
2 & blip & 3 \\
\hline
3 & bass & 39 \\
\hline
4 & deep & 34 \\
\hline
5 & dark & 4\\
\hline
6 & kick & 33\\
\hline
7 & tight & 9\\
\hline
8 & bass lead & 33\\
\hline
9 & empty & 7\\
\hline
10 & tiro & 5\\
\hline
11 & female & 51\\
\hline
12 & house & 59\\
\hline
13 & happy & 4\\
\hline
14 & piano & 11\\
\hline
15 & male & 19\\
\hline
16 & spoken & 7 \\
\hline
17 & aggressive & 8\\
\hline
18 & tztz & 31\\
\hline
19 & disco & 8\\
\hline
20 & chic & 3\\
\hline
21 & melody & 11\\
\hline
\end{tabular}
\quad
\begin{tabular}{| p{.05\textwidth} | p{.20\textwidth} | p{.05\textwidth}|}
\hline
\textbf{No.} & \textbf{Tag} & \textbf{Freq}\\
\hline
22 & jazzy & 5\\
\hline
23 & drums & 4\\
\hline
24 & train & 4\\
\hline
25 & trip & 2\\
\hline
26 & soulful & 27\\
\hline
27 & unzug & 12\\
\hline
28 & song & 32\\
\hline
29 & steady & 12\\
\hline
30 & slow & 8\\
\hline
31 & loopy & 4\\
\hline
32 & volvo & 11\\
\hline
33 & open & 7\\
\hline
34 & strings & 18\\
\hline
35 & techno & 5\\
\hline
36 & tribal & 3\\
\hline
37 & electro & 4\\
\hline
38 & 1 by 2 & 5\\
\hline
39 & flute & 4\\
\hline
40 & gay & 3\\
\hline
41 & gentle & 5\\
\hline
42 & mellow & 4\\
\hline
\end{tabular}
\quad
\begin{tabular}{| p{.05\textwidth} | p{.20\textwidth} | p{.05\textwidth}|}
\hline
\textbf{No.} & \textbf{Tag} & \textbf{Freq}\\
\hline
43 & soft & 2\\
\hline
44 & rough & 2\\
\hline
45 & energy & 8\\
\hline
46 & voices & 3\\
\hline
47 & get down & 6\\
\hline
48 & warm & 4\\
\hline
49 & elegant & 4\\
\hline
50 & funkyg & 6\\
\hline
51 & upbeat & 6\\
\hline
52 & hypnotic & 5\\
\hline 
53 & choir & 3\\
\hline
54 & scary & 3\\
\hline 
55 & horns & 2\\
\hline
56 & drama & 8\\
\hline
57 & space & 6\\
\hline
58 & ethereal & 6\\
\hline
59 & silly & 5\\
\hline
60 & effects & 2\\
\hline
61 & epic & 4\\
\hline
62 & keys & 4\\
\hline
63 & fast & 4\\
\hline
\end{tabular}
\clearpage

\section{Validation Set}
\label{validationset}
\begin{longtable}{| p{.05\textwidth} | p{.95\textwidth} |}
\hline 
1 & v{\_}Marie St James - Closer I Get \\
\hline
2 & v{\_}Degrees Of Motion - Do You Want It Right Now (Extended Club Mix) \\
\hline
3 & v{\_}Alison Limerick - Where Love Lives (Sauna Mix) \\
\hline
4 & v{\_}Soldiers Of Twilight - Believe (Extended) \\
\hline
5 & v{\_}Dj Spiller - Groovejet (If This Ain't Love) (Spiller's Radio Edit) \\
\hline
6 & Soul Providers - Rise (Ricky Montanari {\&} Stefano Greppi Dark Vocal Mix) \\
\hline
7 & v{\_}Filippo Naugthy Moscatello - Final Signal \\
\hline
8 & v{\_}Aaron-Carl - Dance Naked (Soulman's Eviction Loops) \\
\hline
9 & v{\_}Basement Jaxx - Rendez-Vu (All U Crazies) \\
\hline
10 & v{\_}Deep Dish - My Only Sin \\
\hline
11 & v{\_}St Etienne - Only Love Can Break Your Heart (MAW Dub) \\
\hline
12 & v{\_}Mondo Grosso - Star Suite (Shelter Dub) \\
\hline
13 & v{\_}Brother Of Soul - Ife Bobowa \\
\hline
14 & Malena Perez - Cool Lil Thing (Alix Alvarez dub)\\
\hline
15 & v{\_}Big Moses ft Kenny Bobian - Brighter Days (Original Mix)\\
\hline
16 & v{\_}The Product G{\&}B ft Carlos Santana - Dirty Dancing (Robbie Rivera's Tribal Sessions Sub)\\
\hline
17 & v{\_}NSK Tune - Out Of My Mind (NR Main Dub)\\
\hline
18 & v{\_}Johnick - The Captain\\
\hline
19 & v{\_}Mongobonix - Mas-Pito \\
\hline
20 & v{\_}Basement Jaxx - Rendez-Vu (Radio Edit)\\
\hline
21 & v{\_}Mang'e Le Funk - I Still Want You\\
\hline
22 & LUPO - Hell Or Heaven (Extended Mix)\\
\hline
23 & v{\_}Cevin Fisher - Music Saved My Life (Freezy Jam Mix)\\
\hline
24 & v{\_}Capricorn - Love In London (2001 Mix)\\
\hline 
25 & v{\_}Didier Sinclair - Lovely Flight (Dj Chris Pi Airlines Mix)\\
\hline
26 & v{\_}Africanism - Bisou Sucre\\
\hline
27 & v{\_}Freestyle Orchestra - Twi-Lite (K-Dope Twi-Lite)\\
\hline
28 & v{\_}D.J Spen - Da Changez\\
\hline
29 & v{\_}Men From The Nile - Watch Them Come\\
\hline
30 & v{\_}Deepswing - In The Music (Original Version)\\
\hline
31 & Simone - Hey Fellas (Morel's Gospel Mix)\\
\hline
32 & v{\_}Liquid Measures ft Jocelyn Brown - Take Me Up (Pasta Boys club remix)\\
\hline
33 & v{\_}Jaydee - Plastic Dreams (Morales Club Mix)\\
\hline
34 & v{\_}H20 ft Billie - Take Me Higher (Original Mix)\\
\hline
35 & v{\_}Clivilles {\&} Cole - A Deeper Love (A Deeper Feeling Mix)\\
\hline
36 & v{\_}Confession - I Found My Love (Club Mix A)\\
\hline
37 & v{\_}Discorosso - Discorosso (Kama Kama Vrs.No.1)\\
\hline 
38 & v{\_}Depeche Mode - I Feel Loved (Danny Tenaglia's Labor of Love Instrumental)\\
\hline
39 & v{\_}Djum Djum - Difference (Sagwa Mix)\\
\hline
40 & v{\_}Coco Steel {\&} Lovebomb - Feel It (Vox Mix)\\
\hline
41 & v{\_}Coco Steel {\&} Lovebomb - Feel It (Version Mix)\\
\hline
42 & v{\_}Robbie Rivera - Feel This (Robbie Rivera's Tribal Sessions Mix)\\
\hline
43 & v{\_}Paperclip People - Throw\\
\hline
44 & v{\_}MAW - Unreleased Project (Clouds)\\
\hline
45 & v{\_}D.J Spen - Back When\\
\hline
46 & v{\_}E-Smoove - Be With You (Vocal Vibes)\\
\hline
47 & v{\_}Soul Vision - Don't Stop (Edit)\\
\hline 
48 & v{\_}Lil' Louis {\&} The World - Club Lonely (DJ Pierre's Afro Club Mix)\\
\hline
49 & v{\_}Meli'sa Morgan - Still In Love With You (Still In Love Mix)\\
\hline
50 & v{\_}Rachid - Pride (Album Version)\\
\hline
51 & v{\_}Metro Area - Strut\\
\hline
52 & v{\_}Mekon ft Mark Almond - Please Stay (Bertrand Burgalat's Vocal Remix)\\
\hline
53 & v{\_}Massive Attack - Unfinished Sympathy (Paul Oakenfold Mix)\\
\hline
54 & v{\_}Interfront - Bases Mix R-5\\
\hline
55 & v{\_}Interfront - Mathausen\\
\hline 
56 & v{\_}Interfront - Bonus Beats Strange\\
\hline
57 & v{\_}Future Sound Of London - Papua New Guinea (Qube Mix)\\
\hline 
58 & v{\_}Junior Sanchez - N.A.S.T.Y\\
\hline
59 & v{\_}Full Intention pres Hustle Espanol - (Do The) Spanish Hustle (Dub Mix)\\
\hline
60 & v{\_}Brother Of Soul - Celebration Of Life\\
\hline
61 & v{\_}Gat Decor - Passion (D.Emerson Edit)\\
\hline 
62 & v{\_}Paperclip People - Remake Uno\\
\hline 
63 & v{\_}KenLou 3 - What A Sensation (Sensational Beats)\\
\hline
64 & v{\_}Chiapet - Tick Tock (Apocalypse Now Mix)\\
\hline
65 & v{\_}Aaron-Carl - Dance Naked (C{\&}M Balearic Drama Remix)\\
\hline
66 & v{\_}Starchaser - Jambe Myth (Dsnny Jc vs Speed City Mix)\\
\hline
67 & v{\_}Lil' Devious - Come Home (Dave Clarke Remix)\\
\hline 
68 & v{\_}Bob Marley vs Funkstar De Luxe - Sun Is Shining (ATB Club Mix)\\
\hline
69 & v{\_}Cricco Castelli - Life Has Changed\\
\hline
70 & v{\_}Sandy Rivera - Love For Free\\
\hline
71 & Martha Wash - Leave a Light On\\
\hline
72 & v{\_}Lil' Louis {\&} The World - Club Lonely (Bellbottoms {\&} Platforms Mix)\\
\hline
73 & v{\_}Liquid Woman - Come And Go With Me (Anthem Radio Edit)\\
\hline
74 & v{\_}Age Of Love - The Age Of Love (Sign Of The Time Mix)\\
\hline
75 & v{\_}Interfront - Strange Instru\\
\hline
76 & v{\_}Sonique - Feels So Good (En-Motion Remix)\\
\hline
77 & v{\_}Robert Owens - I'll Be Your Friend (Dead Zone)\\
\hline
78 & v{\_}Crystal Waters - Gypsy Woman (Basement Boys 'Strip To The Bone' Mix)\\
\hline
79 & v{\_}Daft Punk - Around The World (Kenlou Mix)\\
\hline
90 & v{\_}Those Guys - Sierra Leone (Main)\\
\hline
81 & v{\_}Kerri Chandler - Atmosphere (Kerri's Foundation Dub)\\
\hline
82 & v{\_}Rui Da Silva - Touch Me (Peace Division Mix)\\
\hline
83 & v{\_}Cottage Alert - Lost Love (Original Demo Cut)\\
\hline
84 & v{\_}Black Masses - Wonderful Person (Wonderful Brazil Dub)\\
\hline
85 & Basil - City Streets (Kerri Chandler Kaoz On City Streets)\\
\hline
86 & v{\_}Africanism - The Dragon\\
\hline
87 & v{\_}Black Science Orchestra - Keep On Keepin' On (Spen {\&} Karisma's Deepah Dub)\\
\hline
88 & v{\_}The Rolling Stones - Saint Of Me (Deep Dish Grunge Garage Dub)\\
\hline
89 & v{\_}Chez Damier - Can You Feel It (Vocal Club Mix)\\
\hline 
90 & v{\_}Armand van Helden - Entra Mi Casa (Original Long Version)\\
\hline
91 & Jamie Lewis ft Michael Watford - For You (Moodbangers Mix)\\
\hline 
92 & KC Flightt vs Funky Junction - Voices (Pete Heller Main Mix)\\
\hline
93 & v{\_}Toni Braxton - Spanish Guitar (Joe Claussel Dub)\\
\hline 
94 & v{\_}Backroom Congregation - Sunday Morning (WMC 98 Instrumental)\\
\hline
95 & Una Mas - I Will Follow You (Full Intention Club Mix)\\
\hline
96 & v{\_}Modjo - No More Tears (Highpass vs Triple X Remix)\\
\hline
97 & v{\_}Massive Attack - Unfinished Sympathy\\
\hline
98 & v{\_}Bob Sinclair - New York City Music\\
\hline
99 & Kimara Lovelace - I Luv You More (Sean McCabe Demo Mix)\\
\hline
100 & v{\_}The Black Science Orchestra - Sunshine (Sunset Vocal)\\
\hline
\end{longtable}

\end{appendices}

%\chapter{Introduction}
%
%\section{Related work}
%
%Use of references: \cite{BTO,Eigen3}
%
%\section{Outline of the thesis}
%
%\chapter{Chapter title}
%
%\chapter{Experimental results}
%
%\chapter{Conclusions and future work}


%\printbibliography[
%heading=bibintoc,
%title={Bibliography}
%]
 
\printglossaries
\chapter*{Bibliography}
\addcontentsline{toc}{chapter}{Bibliography}
\printbibliography[type={inproceedings},title={Proceedings},heading=subbibliography]
\printbibliography[type={article},title={Articles},heading=subbibliography]
\printbibliography[keyword={preprint},title={Pre-Prints},heading=subbibliography]
\printbibliography[type={book},title={Books},heading=subbibliography]
\printbibliography[type={misc},title={Misc},heading=subbibliography]

\end{document}
